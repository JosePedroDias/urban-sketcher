% contexto
The authors are committed to develop a system with the purpose of creating 3 dimensional drafts of
architecture designs using currently available technology and multimodal interfaces.
To build such a system there are several problems that need to be solved.
The authors identify these problems and list possible solutions for them.
A comparative analysis of available systems for architecture creation is conducted.
Finally, conclusions are stated and directions for future work in this area defined.

% problemas
Building such a system involves solving the following problems:
one must find a useful representation for the urban scenery,
offer a proper input interface for the users,
allowing a simple way of sketching
geometry with an easy learning curve, helpful user aids via suggestions and a fluid and efficient
navigation interface.

% solu��es existentes
Several solutions for the problems stated above are reviewed.

% compara��o
A comparative analysis is performed between available products available in the market
which architects use for building design.

% conclus�o
The document ends with a list of conclusions related to the above topics.


\textbf{Keywords:} urban, city, sketch, model, architecture, building

%\hrule