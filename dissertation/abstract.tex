\chapter*{Abstract}
\addcontentsline{toc}{chapter}{Abstract}

%\TODO{review and summarize this part}
%\TODO{CHECK FOR MAXIMUM 250 WORDS!}

\TODO{REVIEW}


% problem

A system is developed for urban scenarios creation um large display screens using laser
pointers as input devices, supporting collaborative usage.
A novel interface is devised to support interaction with such input devices,
based on the concept of gates and circular menus with a non-intrusive interface.


% implementation / features

\TODO{MENU/GATE INTERFACE}
Users can navigate on the virtual world using a set of comprehensive navigation modes
-- first person, bird's eye view and examine, along with a multimodal hand-driven flight mode.
Simple shapes can be created and modeled using straightforward move, extrude, split operations.
Buildings can be instantiated from a library of facade styles by specifying
the desired blueprint and height to generate a unique building.
Objects in the scene can have notes attached to them to support reviewing sessions.

In order to augment the available building possibilities, styles are defined using a dedicated XML grammar
specifying the walls and roof types along with the spatial distribution of the facade attachments.

Facade attachments such as doors, windows and balconies are described in
a simple XML mesh format and an exporter plug in for the Blender 3D modeling software was developed
for conveniently exporting 3D content.


% results

\TODO{TESTS AND CONCLUSION}



%The context and objectives of the project are defined at the introduction.
%Related work is then commented, followed by the analysis of the problem.
%Concepts, tasks and interface are then designed, with their implementation following.
%Evaluation of the system and conclusions end the paper.

%\TODO{CHECK FOR 4 -- 6 KEYWORDS!}
\textbf{Keywords:} architecture prototype, sketch, building, multimodal interface, navigation, reviewing
