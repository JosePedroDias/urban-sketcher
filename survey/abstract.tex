% context and problem
Nowadays an architecture project starts with the architects drafting sketches of building designs
prior to drawing the final CAD documents in order to study viable designs and validating them with clients.
This is currently done with all participants on location, using pen and paper.
The architecture project workflow is highly segmented between the creative part -- without computer intervention --
and the CAD modeling part -- a totally different stage when desktop computers and WIMP interfaces are used.

% motivation
There's an increasing availability of alternative methods for both sketching, navigation in visualization
of 3D scenes.
We're committed to developing a multimodal system allowing multi-user scene navigation, building design
and content reviewing.
It should be possible for the users to interact either locally or from a remote location, using tablet PCs
or other devices facing a powerwall.
It is expected to make part of the architect workflow as a creative and reviewing tool.

% document organization
%This document describes a survey conducted to support the project.
This survey begins with an introduction of the problem.
Projects with similar goals are then analyzed.
It then focuses on the different aspects the project has to face in order to fulfill its goals, namely:
how to get input from users,
how to display interactive content to the users,
shape creation and transformation,
navigation in the scene and
content reviewing procedures.

A comparative analysis between architecture creation systems available in the market is then conducted.
Finally, conclusions are reached and directions for project execution defined.
%future work in this area defined.

%Architects need to draft sketches of building designs prior to drawing
%the final CAD documents in order to study viable designs and discuss them with clients.
%This is currently done with pen and paper. One could offer architects an alternative
%method to draw their sketches, one that models the buildings' drafts in 3D, allowing
%editing shapes, navigating the scene and manipulating additional content,
%such as annotations.

% purpose of the survey
%The author is committed to develop such a system using currently available technology
%and multimodal interfaces.

% summary of sections
%Several problems are identified and possible solutions for them listed.
%A comparative analysis between architecture creation systems available in the market
%is then conducted.
%Finally, conclusions are reached and directions for future work in this area defined.

\textbf{Keywords:} urban architecture, city, building, multimodal interaction, navigation, content reviewing

%\hrule