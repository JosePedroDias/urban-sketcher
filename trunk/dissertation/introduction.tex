\chapter{Introduction}


%\TODO{LESS PARAGRAPHS}
%\TODO{WHY IS IT AN IMPORTANT PROBLEM, WORTHY OF A MsD (NOT SOLVED). DESCRIBE BROAD VIEW OF SOLUTION.}
%\TODO{IMPROVE/TECH STUFF TO ARCHITECTURE OR OUT}
\TODO{AT THE END: CONTRIBUTION AND RESULTS SUMMARIZED}

%%%%%%%%%%%%%%%%%%%

%\section{Context}

% need

With the advent of advanced visualization hardware it is now possible
to interact with complex representations of urban scenarios.
Systems capable of editing 3-dimensional content
tend to be overly complex and make use of concepts
focused on mouse and keyboard interaction.
Fields so disparate as architecture and the entertainment industry
urge to have ways to display and allow multi-user interaction with large screens.


% difficulty

Developing an application for such purposes requires handling several aspects
-- the large scale rendering of the scenario,
giving people means to interact with the screen,
allowing people to interact at the same time and
offering a simple interface adapted to the running environment.


% detail - laser / menu / gate interface

A different input device must be used, both portable and light.
The concurrent input of several users must be handled so each user can
take the most from the large screen area.
By having chosen laser pointers for user input issues arise.
%\TODO{REFERENCE}
A laser pointer can't be tracked while the light is not turned on,
so any clicking behaviors are hard to mimic, even because users can't be 
precise while unable to view the laser pointer's projection on the screen.
So, instead of the commonly used buttons and drop-down menus,
gates and ring menus are introduced.
To solve the screen real estate problem, menus can be invoked as needed
by performing preset strokes on the screen (gestures).
In order to make use of the designed interface, an application with the
purpose of supporting the straightforward creation, navigation and reviewing of urban scenarios is designed.


% detail - common architecture creation workflow

The current most common work flow process for defining the location and overall shape
of a new building is a process composed up of several discrete steps.
First the architect draws rough sketches of the desired building.
These are highly subjective and clients have a hard time in their interpretation and validation.
The architect proceeds on studying the best location and orientation for the building.
This step used to be performed based on 2D maps and on-site inspection.
Once the location and overall shape are defined, a rigorous project is defined using
a Computer Aided Design (CAD) software such as Autodesk Autocad\cite{SITE-AUTOCAD}.
This design is performed from the ground up.
Then if architects want to showcase the building appearance they often rely of rendered images
and pre-rendered animations.


% high-level description of solution

The proposed solution defines a system which allows architects to draft designs
directly on the 3D scenery and allowing clients to better perceive the project at early stages,
giving them navigation and reviewing capabilities.
To accurately merge new buildings on the scenery one needs to obtain
satellite imagery and height maps of the landscape, a feasible task at present times.
Additional benefits could come later on by sharing the landscape information with the CAD
software and eventually exporting the draft buildings for reference.
Another benefit comes from getting a showcase platform for the project set up for free.
Instead of, or additionally to the set of pre-rendered images and short video clips,
clients now have at their disposal a real-time navigation platform which can be used
as a selling ad.


% novelty

This project is therefore a novel approach due to taking advantage of large screen displays
made possible by the devised interface based on gate activation and ring menus,
with users applying discrete strokes using laser pointers.
The navigation and shape creation functionalities are focused on simplicity and a set
of tools is set so additional scenarios and facade styles can be implemented.
A new work flow is proposed so the system is used both at early design stages and
at showcasing the project.

The success of the project shall be measured against the fulfilling of the following goals:


\TODO{INSERT RESULT CONCLUSIONS HERE?}

\begin{description}
	\item[Good interface design] --
		The system works on large display screens using mostly laser pointers
		-- an additional voice command and arm gesture interface is offered for navigation.
		The average user has no experience with these environments.
		The system ought to be driven by first-time users after a few minutes and overcome most of the
		problems derived from the input devices.
		
	\item[Expressiveness of the system] --
		The project shall give users capabilities so simple approximations can be generated.
		Tasks such as navigating the scene or attaching notes to objects ought to be of simple execution.
		Clients might also do small manipulations on the geometry or positioning of objects.
	
	\item[Filling the gap between architect and client] --
		Clients aren't used to visualizing the implicit model by the interpretation of 
		common architectural documents such as orthogonal views or cuts.
		They would surely benefit from 3D virtual approximations.
\end{description}

The system will be subject to usability tests in order to evaluate the fulfillment of these goals.

%\TODO{IS THIS REQUIRED? read all chapters and summarize them}
