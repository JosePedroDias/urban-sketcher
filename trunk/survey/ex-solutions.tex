% EXISTING SOLUTIONS
% seek existing solutions?

%\TODO{arvika}
%\TODO{epoch}
%\TODO{search 'vr authoring frameworks/building'}
%\TODO{ideas from other software - sketchup...}

% TODO 

\subsubsection{ARVIKA, 2003}

ARVIKA \cite{ARVIKA} is a project with sponsoring from the
German Federal Ministry of Education and Research that took place between 1999 and 2003.
It focused on the development of Augmented Reality technologies to aid in industrial tasks.
The consortium involved several industrial partners, such as Volkswagen, BMW, Siemens and Airbus,
the Fraunhofer for Computer Graphics ZGDV institution and the universities of Munich and Aachen.

An expert in the industrial area carried a head mounted display with a camera mounted in it.
The real-time captured video was then interpreted and prepared markers extracted.
The camera's position and orientation was estimated and the head mounted display view was
enriched with virtual objects.

% TODO INSERT PICTURE

The framework was distributed in the form of an ActiveX plug-in for the Internet Explorer Browser
named ARBrowser.

\paragraph{Discussion}
\label{sec:Discussion}

Weidenhausen et al. \cite{ARVIKA-LESSONS} present the deployment of the project as an ActiveX component
to be an advantage since it is based on a widespread program (Internet Explorer) and allowed developers
to create task scenarios with JavaScript and HTML. This easiness is questionable 

Although the world's largest research project in the area, ARVIKA focused too
much on the technical problems regarding AR and little effort was spent on coming up with a
suitable user interface. The authors agree that most people judge the usefulness of a technology
mainly by its user interface so this particular topic became work for future projects. %ARGH!

ARVIKA was meant to support many industrial scenarios -- development, production and service for several
industrial partners in different domains. Creating a scenario was a time consuming -- several days,
according to Weidenhausen et al. -- and required extensive knowledge in 3D modeling tools and VRML.
No authoring capabilities were therefore given to end-users.
This problem was identified as paramount and an authoring tool was scheduled for development,
supporting generic task creation with parameters controlled by the users.
