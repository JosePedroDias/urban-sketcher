%Descri��o sum�ria do que motivou o trabalho efectuado at� agora. 
%Uma descri��o do trabalho realizado at� agora:
%  An�lise de tarefas,
%  prot�tipos de baixa fidelidade,
%  avalia��es heur�sticas,
%  prot�tipos, etc.
%Em resumo, aqui � onde voc�s mostram que j� trabalharam e
%onde apresentam os principais resultados obtidos. 
%(no m�ximo 1,5 p�ginas).

Este projecto � motivado pela vontade de auxiliar uma profiss�o de import�ncia como � a arquitectura.
A inclus�o do projecto no cons�rcio Improve � igualmente motivante, estabelecendo um conjunto
de objectivos e prazos adicionais ao projecto.

O projecto teve in�cio no final de Setembro.
Houve uma reuni�o com o cons�rcio em Lisboa, no INESC Lisboa a que o autor compareceu
e onde foi introduzido aos problemas de desenvolvimento e prazos estipulados.

Houve igualmente um conjunto de reuni�es com vista a elucidar os diversos alunos de mestrado quanto
�s tarefas que deles se espera e os prazos a cumprir.

Foi criado um blog \cite{SITE-BLOG} para documentar o desenvolvimento do projecto,
de modo a fornecer aos orientadores uma ferramenta que lhes permita acompanhar o mesmo,
assim como documentar as diversas etapas para uso do autor.

Seguiu-se uma fase de prepara��o para as ferramentas a usar.
Foram executados pequenos programas para ganhar � vontade no uso do OpenSG \cite{SITE-OPENSG}
e instalado todo o software necess�rio ao desenvolvimento do projecto, entre ele:

\begin{description}
	\item[Microsoft Visual Studio 2003]
		como ambiente de compila��o;
	\item[TortoiseSVN]
		para uso de reposit�rio, permitindo o desenvolvimento colaborativo de software;
	\item[Framework AICI]
		conjunto de classes sobre o qual assenta o Improve e que � baseado em OpenSG
\end{description}

Foram efectuadas diversas reuni�es com o prop�sito de definir objectivos para o projecto
e foram escritos casos de uso.
