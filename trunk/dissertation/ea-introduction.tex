\section{Introduction}


% problem

With the advent of advanced visualization hardware it is now possible
to interact with complex representations of urban scenarios.
Systems capable of editing three-dimensional content
tend to be overly complex and make use of concepts
focused on mouse and keyboard interaction.
There's a demand for systems capable of offering 3D scenes rendering
and supporting multi-user interaction on large screens on fields
so disparate as architecture and the entertainment industry.

The current most common work flow process for defining the location and overall shape
of a new building is a process comprised of several discrete steps.
The architect starts by drawing rough sketches of the desired building.
These are highly subjective, with clients having a hard time in their interpretation and validation.
The architect then proceeds on studying the best location and orientation for the building.
This step used to be performed by observing 2D maps and on-site examination.
Once the location and overall building shape are defined, a rigorous project is drawn using
Computer Aided Design (CAD) software such as Autodesk Autocad\cite{SITE-AUTOCAD}.
This design is performed from scratch.
If architects are required to subsequently showcase the building appearance,
they rely on pre-rendered images and animations.


% related work intro

The state of the art in this domain was sought out. No project mapping the objectives of this project
has been developed though some, due to their dimension or common ground, were subject of analysis.
Work from other authors was reviewed when it addressed relevant problems, subject of usage on the project.
The most well known software bundles were compared to analyze possible solutions and avoid common mistakes.
A set of two interviews served as starting point for obtaining of both the current and desirable work flows
for cityscape creation, along with the definition of the main concepts the system was to support.


% challenge

Developing an application for such purposes requires dealing with several aspects
-- the large scale rendering of the scenario,
giving people means to interact with the screen,
allowing people to interact at the same time and
offering a simple interface adapted to the running environment.


% detail - laser / menu / gate interface

%The concurrent input of several users was handled so that each user
%could make the best possible use of the large screen area.

This system's interface was set for large screen displays and laser pointers chosen as the main source of user input
due to it being both portable and light.
The problem is that a laser pointer can't be tracked while the light is not turned on,
so any clicking behaviors are hard to mimic, even because users can't be 
precise while unable to view the laser pointer's projection on the screen.
An alternate interface urged to solve this issue, so instead of the commonly used buttons and drop-down menus,
crossable areas and ring menus were introduced.
To better explore the available space on the screen,
menus can be invoked as needed by performing preset strokes on the screen (gestures).

% important results

For every mappable feature, tasks were performed on both Urban Sketcher and other system - Google SketchUp.
Even though GSU relies on desktop concepts and input/output devices, users were able to perform the tasks
successfully and took less than twice the time performing them using such devices and interface.
Users easily learned the system's features and their menus, being able to easily
create 3D content such as buildings (template-based) and custom objects (using a tool set of simple modeling tools).


% main contributions

This project takes advantage of large screen displays,
made possible by the devised interface based on area activation and ring menus,
with users applying discrete strokes using laser pointers.
The navigation and shape creation functionalities focused on simplicity,
offering a set of tools and the capability to extend the software with additional scenarios and facade styles.
A novel set of modeling tools and interaction gestures was put to work, with emphasis on a multimodal flight navigation mode.
A new work flow was proposed so the system to be used both at early design stages and
when showcasing the project.


% mapa das estradas

%Following is the related work chapter where three existing solutions are analyzed due to their relevancy,
%along with a set of approaches other authors defined with might suit the application.
%A comparative analysis on commercial software bundles is performed to emphasize their advantages and avoid
%their flaws. The chapter ends with a summary of the obtained data after two interviews with architects.
%The design chapter is next, describing the broad view of the Urban Sketcher project, its architecture, modules
%and purpose. The concepts used throughout this document are defined here.
%Subsequently one can find the implementation chapter, where the various navigation modes are defined and
%justified, as are the content creation, editing and review features which make part of the system.
%A proposed work flow taking advantage of the system developed is then stated.
%The process of evaluation is described, presented and discussed on the evaluation chapter.
%This document ends stating the results which have been reached, the main contributions
%this project introduces and ways it can be enriched in the future.
