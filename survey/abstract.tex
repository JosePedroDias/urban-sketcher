% context and problem
Nowadays the architectural project workflow is highly segmented
between the creative part and the CAD modeling part.
It starts with the architects drafting hand-made sketches of building designs
to study viable designs and validating them with clients, a stage without computer usage, 
with all participants on location using pen and paper.
Later on a set of documents is generated from the ground up on CAD software, 
using desktop computers and WIMP interfaces.

% motivation, goals
There's an increasing availability of alternative methods for both sketching, navigation and visualization
of 3D scenes which can enrich the design process and provide a better experience for both
building drafting, visualization and review.
We're committed to developing a multimodal distributed system to fulfill these goals
using tablet PCs or other devices facing a powerwall.
%We're committed to developing a multimodal system allowing multi-user scene navigation,
%building draft building designs, navigation and content reviewing.
It is thought out to integrate the architectural workflow as a creative and reviewing tool
to complement the early stages of architectural design.

% document organization
This survey begins with an analysis of projects with similar goals,
followed by a set of issues to handle:
how to get input from users,
display interactive content to them,
make shape creation and transformation possible and
allow scene navigation
%provide content reviewing capabilities.
A comparative analysis between architecture creation systems available in the market is then conducted and
finally conclusions are reached and directions for project execution defined.

\textbf{Keywords:} urban architecture, city, building, multimodal interaction, navigation, content reviewing

%\hrule