% INPUT MODALITIES
% handling a laser pointer
% using a tablet pen
% using motion tracking
% issuing voice commands

\TODO{intro}

%\TODO{handling a laser pointer}
%\subsubsection{Laser Pointer}

%\TODO{using a tablet pen}
%\subsubsection{Tablet Pen}


%\TODO{using motion tracking}
\subsection{Input Modalities}

\subsubsection{Motion Tracking}

%Reichner et al. inspected 2 A
%suggest a generalized architecture for
%Augmented Reality applications.
%
%
%\cite{AR-ARCH}


Welch and Foxlin \cite{MT-BULLET} conducted a survey on motion tracking systems,
comparing each solution in terms of cost, precision and capacity to solve the tracking problem.
A group of purposes were identified as main motion tracking applications:
view control, navigation, object selection or manipulation, instrument tracking and avatar animation.

There are motion tracking systems based on measurements of mechanical, inertial, acoustic, magnetic,
optical and radio frequency sensors, each approach bearing its advantages and limitations.

The most robust solution is the combination of two technologies, such as a hybrid between
inertial and acoustic sensors -- the former providing 6 degrees of freedom data and the latter
reading precise positioning for each artifact.

\TODO{BRIDGE WITH STT SYSTEM?}

% TODO more info can be extracted

\paragraph{Discussion}

One can envision the proposed solution to use motion tracking to allow users to change their
point of view in the program, navigate the scene, select and manipulate objects, a subset
of functionality identified by Welch and Foxlin.

\subsubsection{Augmented Reality versus Immersive Virtual Reality}

According to Azuma \cite{OVERVIEW-AR} Augmented Reality should be used
when the collaboration task is co-located,
there's tangible object interaction and
enhanced interaction in the real world.
On the other hand Immersive Virtual Reality is preferred in scenarios for
experiencing the world immersively, that is, in an egocentric perspective;
sharing views and
doing remote collaboration.

\TODO{MORE CONTENT FROM THIS ARTICLE?}

\paragraph{Discussion}
There haven't been set any goals that force the Augmented Reality approach for such a system.
Sharing views and doing remote collaboration are two expected features of this system so
Virtual Reality is the choice to make.


%\TODO{issuing voice commands}
%\subsubsection{Voice Commands}
%
%\cite{SP-GEST-TTOP}