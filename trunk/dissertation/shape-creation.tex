\subsection{Shape Creation}

% SHAPE CREATION AND TRANSFORMATIONS
% for modeling simple buildings
% their translation, scaling, etc.
%\TODO{shape creation}
%\TODO{transformations}

In this section two solutions suitable for conceptual sketching of 3D forms are analyzed.

\subsubsection{SESAME, 2006}

Oh, Stuerzlinger and Danahy \cite{SESAME3D} developed SESAME
(Sketch, Extrude, Sculpt, and Manipulate Easily).
This system tries to provide an interface as powerful and easy to use as 2D sketching on paper.
It is defended by the authors that a 3D model is more easily understood between users
than a regular conceptual design.
It is optimized for modification and allows the creation and editing of volumetric geometry
by extruding 2D contours or sculpting 3D volumes.
It features an enhanced suggestive interface, allowing the creation of lines,
arcs and free-form curves with constraints.
SESAME also supports automatic grouping of objects, i.e., objects related
between themselves (ex: cup on top of table) affect each other.

The user tests conducted comparing SESAME against the popular modeling package
Autodesk 3D Studio Max show that even for experienced 3DSM users
the drawings done with SESAME were more creative and satisfying.

\begin{figure}[!ht]
	\centering
	\includegraphics[width=6cm]{gfx/sesame.png}
	\caption{A view of a drawing done in 40 minutes with SESAME}
	\label{FIG-SESAME}
\end{figure}

\paragraph{Discussion}

This is a promising direction for an urban sketching software to go.
The tests against 3D Studio Max were a bit skewed -- the test should
have been conducted against a system of similar approach, such as Google Sketchup.
The offered interface in SESAME appears poorly thought out.

Some constraints may arise from applying these concepts to a multimodal collaborative
system.



\subsubsection{SmartPaper, 2004}

Shesh and Chen \cite{SMARTPAPER} developed SmartPaper, a system designed to support
2D sketching, featuring oversketching capabilities, sketch on 3D, 3D transforms and
CSG operations. It employs a non-photorealistic rendering technique to convey the
drawing a sketchy look.

\begin{figure}[!ht]
	\centering
	\includegraphics[width=5cm]{gfx/smartpaper.png}
	\caption{Drawing done with SmartPaper}
	\label{FIG-SMARTPAPER}
\end{figure}

\begin{figure}[!ht]
	\centering
	\includegraphics[width=8cm]{gfx/smartpaper2.png}
	\caption{The process of modeling a lamp in SmartPaper}
	\label{FIG-SMARTPAPER2}
\end{figure}

\paragraph{Discussion}

The biggest limitation found in SmartPaper is the lack of curved line support.
Another requirement is that the user must draw all object's edges, not only
the visible ones. In the case of extruded objects this is not problematic, since
the original face would always have to be drawn anyway.
The resulting geometry appears to be irregular but since the goal is to do
conceptual drawings this is not an issue.

\TODOL{RELATE TO OUR PROJECT?}
