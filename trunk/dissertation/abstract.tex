\chapter*{Abstract}
\addcontentsline{toc}{chapter}{Abstract}

%\TODO{review and summarize this part}
%\TODO{CHECK FOR MAXIMUM 250 WORDS!}

A system is developed for large display walls using regular laser pointers
as input devices. It supports collaborative usage and can be deployed
on tablet PCs. A novel interface is devised to support interaction with such
input devices, based on the concept of gates and circular menus
with a non-intrusive interface.

The system supports fast creation of buildings based on strokes to define
its facade boundaries, height and selecting its purpose.

The generation of the building and the populating of its facades with
attachments such as windows, doors and balconies obeys to an XML
grammar that defines the main color theme and texture of the walls,
floor height, facade layout and roof type.
A building purpose is therefore defined by an instance of such grammar,
allowing instancing of buildings such as offices, residential houses and factories.

Architecture projects' locations and overall appearance can be prototyped,
with support for reviewing by attaching notes to 3D objects.
A comprehensive set of navigation modes is offered so
the most common exploration tasks can be easily performed
-- besides first person navigation mode, a bird's eye view,
examine mode and motion tracked navigation modes are offered, with the
latter making use of the user's arms to control flight movement.

The context and objectives of the project are defined at the introduction.
Related work is then commented, followed by the analysis of the problem.
Concepts, tasks and interface are then designed, with their implementation following.
Evaluation of the system and conclusions end the paper.

%\TODO{CHECK FOR 4 -- 6 KEYWORDS!}
\textbf{Keywords:} architecture prototype, sketch, building, multimodal interface, navigation, reviewing
