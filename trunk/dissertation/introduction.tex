\chapter{Introduction}


\TODO{intro to introduction}

%%%%%%%%%%%%%%%%%%%

\section{Context}

Nowadays one has an immense array of input and output devices at disposal,
capable of capturing any sense or property and representing alternate realities
with credibility.

% acabar frase do wimp, set of concepts
Systems dealing heavily with tridimensional content
tend to be overly technical, confined to a set of concepts
and base themselves on a desktop mouse / keyboard / monitor interface
with WIMP.

\TODOL{ADD TYPICAL IMAGE OF 3D SOFTWARE} 

Being an avid user of such systems and someone who enjoys well designed
interfaces, the author became interested in the proposed idea of
creating a multimodal interface for the creation of city sceneries.

% WORKS?!
Having made part of previous projects of less complexity in the college's
multimedia lab and having dealt with rendering 3D content
to a cluster of computers, there was already familiarity with the subject.

% Project project?
% http://urbansketcher.blogspot.com/2006_09_01_archive.html
The fact that the project got to be part of the IMPROVE project,
a consortium sponsored by the european union composed of universities and companies,
only made the objectives more ambitious,
bringing experienced people and their different perspectives to the project.


\subsection{The Architectural Problem}

% survey page 4
Currently an architect starts the building design work flow by drawing
rough free form sketches of the desired building.
These drawings are highly subjective and clients have a hard time in their interpretation and validation.
The information captured in such drawings has little use in the remaining stages of building design.

% REF AUTOCAD GRAPHCAD, etc
Systems such as Autocad \cite{SITE-AUTOCAD} are commonly used for professional building layouts.
The architect replicates his former ideas in the rigid standardized interface offered by these systems.
This is a rigorous endeavor, taking much time to be asserted.

After the main project documents are issued comes the time when the idea has to be pitched to end clients.
In order to gather buyers and investors it's common to generate colorful previews of buildings featuring
increasingly realistic features such as detailed materials, light propagation and crowd simulations.

Despite the ongoing advances in computation, the architectural work flow
keeps these three separated stages that don't share media or applications.
Both the conceptual design and the final reviewing and marketing stages would
benefit from an integrated system with a comprehensive set of design actions
and good navigation and content reviewing capabilities.

The architect can benefit from the system by being able to experiment different
locations for the building and obtain a better integration between the new
design and its surroundings.

\subsection{The Multimedia Lab}

% from rel-int

% projector
% screen -> oco...
LemeWall \cite{LEME} is a computer cluster of computers and projectors, with the projectors
being mounted in a matrix of 4 x 3, composing an image by back-projection on a screen.
The computers form a intranet allowing any computer to control them all and render to the screen.

\TODOL{inserir imagem leme e esquema do cluster}

The lab has a set of infrared cameras mounted on the back that are capable of detecting
laser pointers when pointed at the screen.

\TODOL{Artigo lasers to jota? artigo novo da sala?}

\TODOL{imagem motion tracking / esquema de cams de IMMI}

Additionally motion tracking systems from STT have been used in the lab, for with two and four cameras,
allowing real-time capturing of a set of reflective markers' positions in the area near to the screen.

LemeWall is located at Prof. Louren�o Fernandes Lab, in Instituto Superior T�cnico, Tagus Park, Oeiras.


\subsection{Taking Part in IMPROVE}

% http://urbansketcher.blogspot.com/2006_09_01_archive.html

IMPROVE \cite{SITE-IMPROVE} is a consortium of universities, hardware manufacturers and design studios in
the areas of architecture and car design.
The universities provide the developers and scientific innovation.
The hardware manufacturers come up with prototype hardware devices
to answer the project's needs and  support the developers in using them.
The design studios serve as the main testers of the system, providing
suggestions and feedback to the project so that the final results better suit their needs.

The IMPROVE project is funded by the European Union.
Instituto Superior T�cnico got responsible for creating an application to the architecture scenario
supporting both tablet PC and the power wall.

Although the two scenarios are very different, the applications must share a common functionality subset
(such as displaying the same geometry, providing annotation support and navigation) and a consistent user interface.

\TODOL{image of GIDES?}

Since the beginning of IMPROVE in 2005, IST had been using GIDES++ to test IMPROVE's functionality.
GIDES++ is a CAD/CAM application \cite{GIDES} in orthographic 3D with a non-intrusive interface,
having been developed by several students throughout the last years.

A decision was made that GIDES++ could no longer serve IMPROVE due to its monolithic development and constrained 3D capabilities.

A new project was built from the ground up with the requirements of hardware and interaction in mind,
making use of OpenSG \cite{SITE-OPENSG}, a scene graph system with support for rendering and synchronization between nodes of a cluster, providing other features such as multi-threading.

On top of it an event framework named AICI \cite{AICI}
was used and extended by our team.
AICI is a framework developed both in Germany and South Korea, being part of IMPROVE's consortium.

OSGA \cite{OSGA} is a communication backbone based on XML.
It provides a way of sharing events between several applications through a clean XML schema.
Each application can both subscribe types of events and/or generate them.

%\TODO{TOO MUCH FRAMEWORK DETAILS?}

%%%%%%%%%%%%%%%%%%%

\section{Objectives}

The main purposes of the project are to allow architects to test possible solutions for
building planning and to provide a way for clients and architects to review them.

The project is therefore thought out to be applied in early stages in the conception of
architectural projects for clients to convey the spatial distribution of the proposals in discussion.
The content creator, which is henceforth called the architect, should be given the tools so he
can create facades of buildings and test different building configurations in an experimental fashion.
Creating three dimensional content should be an easy and imprecise task --
it should be considered a rough approximation.

The system may also serve as an interface for clients to explore the
surroundings of the buildings at a later stage,
either to validate the design or for marketing purposes.

It is therefore crucial for a novice user to get the grip of the system easily.
Tasks such as navigating the scene or attaching notes to objects ought to be of simple execution.
Clients might also do small manipulations on the geometry or appearance of objects.

Both scenarios should support users individually working with tablet PCs or
a collaborative scenario in front of a power wall.

The success of this project shall therefore be evaluated by fulfilling the following goals:

\begin{description}
	\item[Expressiveness of the system] --
		The user should be given the expressiveness power of a sketch board.
		There should be common building templates and it should be configurable with a good interface.
	\item[Good interface design] --
		The system will feature a varied set of input interfaces.
		The mixed use of sketching with laser pointer/pen, handling of artifacts and issuing
		voice commands implies an interaction metaphor far from the WIMP interaction,
		by far the most common approach to this problem.
		The effectiveness and efficacy of the user interaction with the system and the
		learning curve will dictate how well the project has succeeded.
	\item[Filling the gap between architect and client] --
		Another goal of the system is to allow the client to navigate throughout the
		architectural model and validate it with small manipulations and note attachments.
		Clients aren't used to visualizing the implicit model by the interpretation of 
		common architectural documents such as orthogonal views or cuts.
		They would surely benefit from 3D virtual approximations.
\end{description}

The system will be subject to usability tests in order to evaluate the fulfillment of these goals.

%%%%%%%%%%%%%%%%%%%

%\section{Thesis Structure}

\TODO{read all chapters and summarize them}
