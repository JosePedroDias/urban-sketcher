\chapter{TEMP Complements TEMP}

\section{Shape Internal Structure}

A shape in this system is a boundary representation (B-REP) \cite{REP-SOLID}.
The shape surface is defined by two data structures:
an indexed array of \textbf{vertex positions}, used to store the position of each shape vertex;
an indexed array of \textbf{faces}, with every face being a counter-clockwise ordered set of 4 vertex indices.
An \textbf{edge} in the system is a pair of ordered vertex indices.
The ordering step makes the edge direction irrelevant for the edge definition, a desired feature.

Besides this information, each shape manages an auxiliary \textbf{edge map}.
This map associates a face to its bounding edges and vice versa.
The edge map allows efficient queries to be performed, such as:
\textit{which faces are bound by edge x};
\textit{what is the opposite edge of edge x in face y};
\textit{which other edges beside edge x make part of the face loop}.
These are all relevant queries for the implementation of internal shape operations.

Shapes offer methods to update their visual representation.
Shapes also make available a set of geometry modifying operations --
these are not only exposed to the user via UI, but also used by templates, as further explained.

In order to support undo operations, the memento design pattern \cite{despat} was implemented.
A shape memento stores the internal structure of a shape at one point in time
and is able to restore that shape's state later on, if so requested.
A shape has a stack of mementos so multiple undo steps can be executed.



\section{Template}

A template is a shape recipe with input parameters, defining a way of obtaining the desired shape
by applying a set of operations, with the passed parameters affecting the final geometry.
Templates are not directly exposed on the system UI. They're currently used solely to generate building ceilings,
but could be extended for other means.

Templates make use of shape exposed operations to generate the final shape.
For instance, the creation of a 2-slanted ceiling starts from a regular box shape,
applying a sequence of 2 edge move operations.
\TODO{EXAMPLE 2-SLANTED CEILING}
Additional templates could be easily created to aid in the generation of repetitive structures
such as railings and staircases.


\section{Building Style Definition}

A building is a complex shape, composed of side walls, a ceiling and a set of attached shapes enriching the fa�ades
with details such as doors, windows and balconies.

A building style is a set of rules and parameters, written according to an XML grammar. 
The system comes with a set of styles, with the user choosing the desired style to apply to the building he's about to create.
The building style factory is able to parse the style definition, instantiating and distributing attached shapes to make up
the fa�ades according to the chosen style.
\TODO{ADD XSD AND REFERENCE IT}


\section{Supported Operations / Relevant Future Work}

This system has the goal of offering an easy yet reasonably powerful interface for modeling shapes.
The shape's internal structure was planned so both face and edge-based operations could be performed.
Every operation takes the triggering element (edge or face) as input parameter.
Most operations require additional information, obtained by extracting the user's stroke direction and length.
This interaction model keeps the number of stroke steps minimal while offering valid functionality for each operation.


\subsection{Object Operations}

Available object operations are \textbf{translation}, \textbf{rotation}, \textbf{scale} and \textbf{clone}.
The \textbf{translation} operation accepts a delta vector, applying it in real-time on one of the 5 directions:
normal and the 4 surrounding edge directions.
\textbf{Rotation} and \textbf{scale} operations take only the stroke length --
scale transforms all 3 axes proportionally;
rotation can take the triggered face's normal as axis of rotation or can default to the YY axis for simplification,
since most urban changing operations make use of this rotation.
An additional object operation is \textbf{cloning}. The clone operation works like a regular translation,
leaving behind a copy of the original shape.
Since it uses the face normal and face-bounding edges' directions, the cloning operation allows efficient generation
of building blocks and repetitive structures.


\subsection{Face Operations}

The \textbf{move} operation uses the same directions as translation, changing the position of the face vertices.
\textbf{Move neighbors} identifies neighboring faces having a smaller angle with the selected face than a defined threshold,
applying a move operation to the set of affected faces.
\textbf{Extrude} generates a new face for every face's edge and offers only the normal direction for moving the face outwards/inwards.
\textbf{Bevel} scales the face vertices, preserving inner angles.
The \textbf{beveled extrude} operation exists solely for convenience, generating a set of faces and applying an immediate bevel operation
to provide a simple way of creating relief details on faces for chimneys or windows.


\subsection{Edge Operations}

Edges can be \textbf{moved}, with the offered directions being the edge normal and the opposite edges along the neighboring faces.
\TODO{EDGE DIRS}.

The \textbf{split along face loop} operation allows increasing the detail of a shape by cutting new faces along the center of the
implied edges.
\TODO{SPLIT}


\subsection{Possible Additions}

The set of given operations offers a competent tool set for modeling simple shapes.
During tests users have found several creative ways of modeling the requested shapes.
Even so, multi-selection operations would be a nice addition to the tool set.
In early prototypes of the system the closed lasso stroke was used to select multiple objects, which proved problematic
since users inadvertently selected too many objects.
Furthermore, using multi-selection for shape operations limited the direction seeking algorithms described above,
which are key to the proposed simplified interface.


\section{Main Menu vs Contextual Menu}

Every action which generates new shapes is accessible from the main menu.
Actions which change existing contents are available from context menus for face and edge.
This segmentation rule was enforced so users know where to search when in need of an untried operation.


\section{Navigation Modes - why, relevant future work}

Good navigation modes are paramount for the success of any 3D based system.
The nature of this system, with the availability of a large screen,
stroke-controlled input and the aim of allowing unexperienced users to take control
of the system quickly, made this requirement even more relevant.

According both to the task at hand and personal taste,
several alternative modes may be useful to users.
For tasks of simulating human movement, a \textbf{first person} based mode is expected.
When searching for scenario features and moving big distances, a \textbf{top-down view} manipulation mode can be of aid.
In occasions when an object is clearly the center of attention of the user,
such as shape exploration and modeling operations, a \textbf{view centered on the object} mode is helpful.
For giving an overview of the scenario and showcasing blocks of buildings, a \textbf{flight movement} mode might be interesting.

The first 3 of the 4 mentioned modes were implemented with regular menu/gate widgets.
The flight mode is multimodal, relying on arm tracking and speech controlled input.


\subsection{1st person}

Similar to most first person shooters, offers translations and rotations around the current point of view (POV).

A helpful addition to this mode would be collision detection, to keep users from crossing walls and the ground.
This could also help users moving on ramps and staircases.
The application of SmartCam \cite{SMARTCAM}, commented on section \ref{SMARTCAM-LABEL} would also enhance this mode,
at least the physics spring model simulation part.


\subsection{Compass}

A birds-eye-view of the POV is helpful in many ways. It allows the user to better perceive his positioning in the scenario.
The superimposed cardinal compass points are useful, particularly for people with a more technical background.
The dragging gesture is increasingly more popular, and this mode uses it extensively:
dragging the center view translates the user along the ground plane;
dragging the outer ring rotates the POV, a concept easily grasped by test users.

To enhance the reach of the translation movement, the Speed-dependent zooming \cite{SPEEDZOOM},
commented on section \ref{SPEEDZOOM-LABEL} could be applied,
translating drag velocity into exponential translation changes.


\subsection{Examine}

The examine mode is based on moving along the space close to the center of attention.
The user is offered a gate so a new center of attention can be set.
Once this is done, a spherical widget allows performing rotations around the object by dragging the sphere.
Two additional gates allow zooming in and out to reveal more or less detail, respectively.

For the users who got the grip of this mode, it has revealed itself a very efficient way for both looking around and repositioning oneself.
Only laser-tracking problems inhibited a better use of repeatedly re-centering operation for movement.


\subsection{Flight}

Given the available space in front of the large screen display and the existence of tracking hardware on the test lab,
an experimental mode was implemented, integrating the input extracted from tracking users arm movements and speech commands.
The idea is of offering a hands-on mode for flying.
The arm movements control both flight speed, rotation and altitude shift.
The voice commands are only required for controlling the application/inhibition of this mode.
...


\subsection{Other navigational modes}

In addition to the navigation modes made available in the system, the following modes might be of use.

The multimodal ``go-to-here'' \TODO{REF}, with a combination of laser pointing to the destination
and speech command to trigger the movement. This mode was requested on the final tests.

The stroke-defined path, as suggested on the Path Drawing \cite{PATH3D}, discussed on section \ref{PATH3D-LABEL}.
It would be useful to experiment this mode, but the requirements for its application were unmatchable:
Igarashi's system uses a third person view with explicit avatar rendering. Moreover,
Urban Sketcher stroke-based interface makes it hard to map a movement path stroke
-- this would require multimodal integration with a ``move-like-this'' speech command.

During the navigation tests at Glasgow, users suggested the addition of a list of recorded locations.
The idea is for the user to get to an unrecorded and relevant point of view, such as a good fa�ade angle,
and trigger the record location action.
Recorded locations might be tagged by either speech recognition (example: ``record location \textbf{stadium front} now'')
or scribble text recognition. Later on resuming to the saved location would be a matter of invoking the related metadata.


\section{Evaluation - intro, objectives, improve, ccu}

The system was periodically tested throughout its design and implementation stages.
Emphasis was put in obtaining a user-centered interface,
thus user interface methodologies and tests were applied.TODO

\TODO{improve}

Objectives

Most of the Urban Sketcher features were tested by handing out a set of vaguely
defined tasks for users to perform.

The largest possible subset of tasks was also performed on a well established software product.
For this matter, Google SketchUp was chosen. GSU also denotes the aim of offering
a simple yet powerful interface. Even so, it is a desktop-based program and heavily relies on WIMP
concepts and mouse input.
Most tasks were successfully reproduced with Google SketchUp, with only the building creation task
absent for lack of supporting tools in the product.

Tasks performed on both systems were expected to show discrepancies in both completion time
and number of errors.
Most users are proefficient working with the mouse.
The \textit{ad-hoc} laser tracking system used in the test:
didn't map the whole screen;
had calibration issues, particularly near the edges;
had a low sampling rate (about 20 Hz).

The point is to try to obtain reasonable times and expect US to overcome
many problems by the convenient grouping of actions in both the main and contextual menus,
along with the alternate method for applying operations on shapes and navigating the scene.
TODO


Evaluation comments

The menu and gate interaction model proved both usable and satisfactory for the tested users.
They benefited from having the interface logically grouped and many gained confidence
while performing operations they'd never tried before and used to find intimidating, such as modeling shapes.

Some navigation modes might gain from having a more tight ``safety net''.
In order to keep users from being let down from a mode and avoiding it thereafter:
some users crossed over walls/the ground;
some users dollied around objects into bad POVs.


\section{Design}

This section introduces the concepts used extensively throughout the system.
It also describes the modules which make part of the system and each module responsibility.

Urban Sketcher is composed of a set of modules, most of them implemented as singleton classes
for managing subsets of the functionality.

A set of input adapters get data from several media -- laser strokes, mouse pointer events,
speech commands and motion tracked markers' coordinates.
These adapters generate commands which are consumed by higher level managers.

\textbf{Strokes} are managed and at low level, with key stroke gestures triggering commands.
A shape recognition engine named \textbf{Cali}\TODO{REF!} was integrated to provide shape detection on 2 cases:
triangle, to invoke the main menu and
rectangle, to identify blueprint sketches over construction planes.
Cali is fed polyline information and returns the recognized shape family and its parameters, if found.
The \textbf{2D Widget Manager} takes care of the visible interface, handling gate and menu events.

The \textbf{Navigation Manager} is responsible for keeping positioning information up to date.
It transforms the positioning state in response to actions issued by the supported navigation modes.

The \textbf{Shape Manager} holds the existing shape references and provides means for them to be selected and manipulated.
It caches loaded shapes to enhance to performance in the generation of buildings, where attached geometry is bound
to repeat.

The \textbf{Building Style Factory} loads and parses style definitions into fa�ade-generating algorithms.
Once fed with blueprint information, desired style and height, this component is able to instantiate
a building and its details.

The \textbf{2D Gate/Menu Interface} is a set of widgets and their logic, affected by strokes and other media-generated commands
such as speech and tracking events.
The widgets make use of the Model-View-Controller design pattern, with the controller ultimately making use of
higher level managers functionalities.
These widgets are oftentimes created by composition of simpler, general purpose widgets.


\section{Implementation}

The execution of the projected features described on the previous section is the focus of Implementation.
Functionalities are grouped by purpose and its details explained and discussed.
To close this section, an alternative work flow is proposed to take advantage of the system features.


Evaluation
2 diagramas:
Lab - 2 cams tracking, 12 proj
Glasgow - 4 cams tracking, 1 proj

Justificar flight n ter sido testado no fim por falta de cams, tempo e mapeamento no GSU.

Rever coment�rio Eval

falar de median pra sample < 30, complementar \% com (a/b)

rever conclus�o



\section{Related Work - why these, summary table, what has been learned}

adicionar instant architecture

ver outros artigos n�o mencionados
