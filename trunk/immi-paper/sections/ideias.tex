\section{Trabalho Futuro}

%Na navega��o tipo "super-homem", o controle da velocidade � efectuado com comandos %de voz (para aumentar ou diminuir). Uma forma alternativa de o fazer �  controlando %a velocidade da proximidade das extremidades dos bra�os do utilizador (leia-se as %m�os) face ao torso do mesmo, dispensando assim a interac��o vocal e obtendo um %controle cont�nuo em vez do discreto 'mais r�pido'/'mais devagar'.

%<INSERIR IMAGEM STOP DE FLY>

Podem ser explorados cen�rios mais complexos com estes modos de interac��o,
como � o caso de jogos l�dicos.

Uma capacidade interessante seria a possibilidade de interac��o multi-utilizador,
para desenvolver ac��es colaborativas, como por exemplo um utilizador navegar na
cena e um segundo utilizador manipular objectos da mesma.

O estado do sistema poderia ser facilmente percept�vel no ecr� pela impress�o de texto/�cone para identificar o presente modo.

Durante as opera��es, poderia opcionalmente surgir informa��o sobre as altera��es produzidas, como a percentagem de escala aplicada,
ilustra��o de eixo e �ngulo de rota��o, etc.

Foi sugerida por um utilizador a faculdade de baptizar posi��es do mapa para posterior navega��o pela sua invoca��o.

A grava��o de um plano de voo poderia igualmente ser disponibilizada, para posterior reprodu��o -- uma esp�cie de modo demonstra��o.
