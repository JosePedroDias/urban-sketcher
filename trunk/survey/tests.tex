\section{Tests}

\subsection{Colors and URLs}
Let's write in red from \textcolor{red}{now on}.\\
We can put a \colorbox{cyan}{color box} too.\\
This is an url: \url{http://urbansketcher.blogspot.com}

\subsection{Footnote}
This is an important cryptic subject\footnote{and this is its explanation}.

\subsection{Bulleted List}
\begin{itemize}
	\item first
	\item second
	\item third
\end{itemize}

\subsection{Numbered List}
\begin{enumerate}
	\item foo
	\item bar
	\item baz
\end{enumerate}

\subsection{Columns}
\begin{tabbing}
	COLUMN1	\=COLUMN2	\=COLUMN3\\
	Col1	\>Col2		\>Col3\\
	Col1	\>Col2		\>Col3
\end{tabbing}

\subsection{Tables}

% l, c, r defines a column and its horizontal formatting
% | for a line between columns
% \hline for a ruler between rows
% & separates cells in the same row
% valores no float:
%	h - here
%	t - ?
%	p - ?
%	b - ?
\begin{table}[htpb]
	\begin{tabular}{|c||cc|}
		\hline
		one & two & three \\
		\hline\hline
		four & five & six\\
		seven & eight & nine\\
		\hline
	\end{tabular}
	\caption{table with a caption}
	\label{lTable1}
\end{table}

\begin{table}[htpb]
	\begin{tabular}{|l|c|r|}
		\hline
		one & \multicolumn{2}{c|}{two \& three} \\
		\hline
		four & five & six\\
		seven & eight & nine\\
		\hline
	\end{tabular}
	\caption{another table with a caption}
\end{table}

Let's reference the first table \ref{lTable1} now

\subsection{References}
Here's a reference \cite{liu01smart}

\subsection{Images}
\begin{figure}[htb]
	\centering
	\includegraphics{gfx/tux.png}
	\caption{Tux, the Linux mascot}
\end{figure}

\begin{figure}[htb]
	\centering
	\includegraphics[scale=0.5]{gfx/platypus.jpg}
	\caption{A platypus}
\end{figure}

%\subsection{Using the Glossary}
%\glossary{EPS}{Encapsulated PostScript}