\chapter{Evaluation}

\TODO{intro to evaluation}


\section{User Tests for IMPROVE}

The inclusion of this project in the IMPROVE initiative provided an opportunity
for the interface and some of its widgets to be tested throughout the development.

Three user tests were carried out:

\begin{table}[ht]
		\centering
				\begin{tabular}{|l|l|l|}
						\hline
						Date						& Location					& Summary \\
						\hline
						16 -- 19 April	& Glasgow, Scotland	& power wall versus tablet PC \\
						11 -- 12 June		& Lisbon, Portugal	& collaboration on power wall \\
						16 -- 19 July		& Glasgow, Scotland	& tracked flight \\ %, voice recognition, object instancing 
						\hline
				\end{tabular}
		\caption{User Tests for IMPROVE}
		\label{tab:user-tests}
\end{table}

\subsection{Evaluated Subjects}

The gate concept, the first person and compass navigation modes were tested extensively,
both in the Glasgow April, Lisbon and Glasgow July tests.
The motion tracked flight mode was tested at the Glasgow July test.

Throughout the tests several enhancements were made regarding the interface.

Gate tooltips were created to aid new users in the understanding of each gate functionality.
Gate icons were validated, with some of them being remade for an easier interpretation by users.

Widget layouts were corrected, with the first person navigation widget's gates being reordered
for better performance, keeping the center area empty to allow a stroke to continuously activate
several gates and grouping opposite actions for fast correction of overdone actions.

The compass navigation widget was simplified -- its first release featured 4 gates for
moving up/down and zooming up/down the center miniature map. These gates used to appear on top
of the cardinal points outer ring, with them disappearing if it was rotated. This proved
confusing, with the additional problem that the move up/down icons -- blue air balloons --
were reported to resemble light bulbs and therefore were not used as regularly as they should at the Glasgow April test.

This test was performed on a hand made wall with a barely planar surface.
The algorithm for laser tracking wasn't as evolved as it is now so the power wall performance
was mediocre when compared to using tablet PCs -- the drawn strokes often appeared centimeters
apart from the laser point that has originated them, a technology problem that has been corrected
and has no direct relationship with this project.

During this test the feedback to the newly created compass navigation widget was enthusiastic.
It was observed that the users with an artistic background used the first person mode oftentimes while the
ones with a more technical background used the compass more often. It provides a faster way
of moving to distant places and is also effective because the surroundings are visible while
the user drags the map.

A selection mode based on picking gave bad results. Users picked objects inadvertently and
sometimes even dragged them, what built up chaos since large objects were displaced, confusing
the user and blocking its vision. These false picks occurred more often using the power wall
because stroke segmenting occurs occasionally there, due mainly to bad laser calibration.
This picking mode was disabled on the last Glasgow user test, with a clear reduction in
badly detected operations.

The motion tracked flight mode was introduced in the last Glasgow test with the 4 cameras
setup described in the section \ref{sec:imp-fly-nav} Fly Navigation Mode, on page \pageref{sec:imp-fly-nav}.
Users had to wear 4 fabric bands with markers attached, closed by velcro.
The most critical problem detected regarding this mode had to do with the detection of the starting pose.
Most users often move their arms enthusiastically while performing others tasks, so when issuing the
``begin flying'' command, the system sometimes interpreted incorrectly which markers belonged where
(a labeling problem). This problem was addressed by asking users to leave the tracked space in the
vicinity of the screen for a short period (10 seconds), returning to it with the arms raised towards
the screen, to aid the motion tracking system to correctly do the labeling.
After a minute of getting used to the three main available interpretable actions, users not only got
to control the flight effectively, but also coupled actions correctly without any supervision
-- such as slowing down and turning or turning and lowering the altitude at the same time --
while visibly enjoying the experience. The shorter and fatter user in the test had serious trouble
with this mode, being unable to start flying correctly. In the other end, tall users controlled the
system perfectly, which leads to the conclusion that marker occlusion problems exist even with
four cameras for subjects with short stature, a technology problem that doesn't compromise the idea
behind this set of gestures and their associated actions.


\section{Dedicated Tests}

\subsection{Building Creation}

This test is addressed for execution during the first weeks of October.

\subsection{Custom Shape modeling versus SketchUp}

This test is addressed for execution during the first weeks of October.
