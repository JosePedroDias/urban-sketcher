\section*{Sketching}



\subsection*{Preserving the Freedom of Sketching\\to Create a Natural Computer-Based Sketch Tool}

\begin{table}[htpb]
    \begin{tabular}{|r|l|}
        \hline
        tag & \cite{FREEDOM01} \\
        authors & Christine Alvarado and Randall Davis \\
        url & http://rationale.csail.mit.edu/publications/Alvarado2001Preserving.pdf \\
        year & 2001 \\
        cited by & 11 \\
        pages & 6 \\
        \hline
    \end{tabular}
\end{table}

%The paper presents the work toward an interface for mechanical designers named ASSIST,
%with the purpose of allowing them to sketch naturally and have the computer to interpret
%their strokes into shapes like rods, hinges, polygons, etc.
%
%The interpretation is a three stages procedure:
%
%\begin{itemize}
%	\item match strokes to a series of templates
%	\item rank interpretations with several heuristics about drawing style and mechanical engineering
%	\item return the most consistent hypothesis
%\end{itemize}
%
%The authors emphasize the difficulty they faced when replacing the original human-made stroke
%with the computer interpretation. Users prefer composing the whole drawing prior to the computer
%interpretation replacement. Having the computer interpreting every stroke makes them feel they're
%giving up control of the program. Even so, that was the path chosen by the authors because for
%every extra stroke without giving feedback to the user the chance of misinterpretations increases.
%Another relevant conclusion is that users expect symmetry to be kept regarding the interpreted
%shapes, so it would be a good idea to detect and suggest alignment restrictions between shapes.



\subsection*{Interactive beautification: a technique for rapid geometric design}

\begin{table}[htpb]
    \begin{tabular}{|r|l|}
        \hline
        tag & \cite{BEAUTY97} \\
        authors & Takeo Igarashi and Satoshi Matsuoka and Sachiko Kawachiya and Hidehiko Tanaka \\
        url & http://www-ui.is.s.u-tokyo.ac.jp/~~takeo/papers/uist97.pdf \\
        %\multicolumn{2}{l r|l r|l r}{year & 1997 & cited by & 87 & pages & 10} \\
        year & 1997 \\
        cited by & 87 \\
        pages & 10 \\
        \hline
    \end{tabular}
\end{table}

%The paper presents a 2D sketching system named PEGASUS. It receives user strokes and converts
%them generating candidates taking into account the restrictions for: vertex connection,
%segment connection, parallelism, perpendicularity, alignment, congruence, symmetry and interval equality.
%The application presents the user the most relevant candidates and highlights the highest relevant one.
%The user can either accept it or select another candidate by tapping on it. The technique appears
%to be of great use in my project - in architecture most walls subsume these constraints.
%Additionally my users will model in 3D, which increases the interest in applying beautification algorithms
%such as this one.



\subsection*{Towards the use of cad models in VR applications}

\begin{table}[htpb]
    \begin{tabular}{|r|l|}
        \hline
        tag & \cite{CADVR06} \\
        authors & Alberto Raposo and Eduardo T. L. Corseuil and \\
        & Gustavo N. Wagner and Ismael H. F. dos Santos and Marcelo Gattass \\
        url & http://www.tecgraf.puc-rio.br/~abraposo/pubs/VRCIA2006/VRCIA2006.zip \\
        %\multicolumn{2}{l r|l r|l r}{year & 1997 & cited by & 87 & pages & 10} \\
        year & 2006 \\
        cited by & not in scholar \\
        pages & 7 \\
        \hline
    \end{tabular}
\end{table}

The authors create a desktop program named ENVIRON and a series of plug ins to common
CAD programs (namely Bentley's MicroStation and Autodesk's 3D Studio Max).

The aim of their work is making the bridge between CAD programs and VR systems.
They point out a series of problems that arise in the conversion process:

%\begin{itemize}
%	\item \textbf{low performance} -- due to unneeded model complexity
%	\item \textbf{lack of realism} -- usually users of CAD programs don't associate material
%	or texture data to objects
%	\item \textbf{inadequate treatment of geometry} -- the conversion often with loss of
%	geometry, precision and errors such as badly oriented normals
%	\item \textbf{loss of semantics} -- names, layers and other structures aren't converted
%	\item \textbf{one way conversion} -- there's no direct way to apply the changes made in the
%	VR system back to the CAD program 
%\end{itemize}

The authors discuss different ways to couple CAD and VR:

\begin{itemize}
	\item gateways through standard format such as VRML
	\item defining a common format such as XMpLant
	\item sharing an API such as OSG CAD
	\item sharing the same core, rarely possible due to vendor specificity and closed APIs
\end{itemize}

Another approach is through the graphics pipeline of the 2 programs, and its 4 levels:
feature modeler, modeler, tesselator and renderer.

The application handles parametric objects, nurbs and triangle meshes. Importing these
types bring additional problems such as T-vertices, which are discussed and a solution
is given.

Since the program is focused on oil platform creation, water and terrain rendering are
relevant and algorithms for rendering these features are presented.

The team developed ViRAL too, a pluggable extension that allows several VR devices and
views to be used with the same base code.



\subsection*{Sketching annotations in a 3D web environment}

\begin{table}[htpb]
    \begin{tabular}{|r|l|}
        \hline
        tag & \cite{SKAN02} \\
        authors & Thomas Jung and Mark D. Gross and Ellen Yi-Luen Do \\
        url & http://code.arc.cmu.edu/dmgftp/publications/pdfs/SpacePenCHI02.pdf \\
        year & 2002 \\
        cited by & 4 \\
        pages & 2 \\
        \hline
    \end{tabular}
\end{table}

Space Pen was developed to fill the needs of architects which are frequently
required to review the design of buildings with clients.

Instead of being constrained to go there and stare at floor plans and sections,
which clients find hard to interpret, Space Pen provides both floor pan view of the 
floor the client is in with him marked as red as an additional first person perspective.

The system allows surface sketching of annotation with optional text attachments.
It also lets a user draw a temporary plane where shapes can be drawn. Rectangles
might be interpreted as either windows or doors.



\subsection*{Smart Sketchpad - An On-line Graphics Recognition System}

\begin{table}[htpb]
    \begin{tabular}{|r|l|}
        \hline
        tag & \cite{SMARTSK01} \\
        authors & Liu Wenyin and Wenjie Qian and Rong Xiao and Xiangyu Jin \\
        url & http://www.cs.cityu.edu.hk/~liuwy/publications/Smartsketchpad\_ICDAR01.pdf \\
        year & 2001 \\
        cited by & either 3 or 16 \\
        pages & 5 \\
        \hline
    \end{tabular}
\end{table}

%Smart Sketchpad recognizes standard shapes (rectangles, triangles, ellipses, straight lines)
%and compound shapes such as arrowheads.
%
%The article describes the steps necessary for shape recognition:
%
%\begin{enumerate}
%	\item input as a chain of points
%	\item polygonalize to polyline and refine endpoints
%	\item close near endpoints - if closed go to step 6
%	\item if line ends near another line end, join them and go to step 3
%	\item classify line as one of: straight line, polyline or free form curve. go to step 8
%	\item close shape recognition *
%	\item estimate the parameter of the closed shape
%	\item test if shape can be combined with other shapes in the drawing. If so repeat step 8, else end
%\end{enumerate}
%
%The $6^{th}$ step was tested with: rule-based system, support vector machine and neural network.
%The most successfully approach was SVN, with 97,5\% success, closely followed by NN.



\subsection*{Building the Design Studio of the Future}

\begin{table}[htpb]
    \begin{tabular}{|r|l|}
        \hline
        tag & \cite{DESFUT04} \\
        authors & Aaron Adler and Jacob Eisenstein and Michael Oltmans \\
        & and Lisa Guttentag and Randall Davis \\
        url & http://www.rationale.csail.mit.edu/publications/Adler2004Building.pdf \\
        year & 2004 \\
        cited by & 3 \\
        pages & 7 \\
        \hline
    \end{tabular}
\end{table}

Adler, Eisenstein et al. present the ongoing work at MIT seeking the most effective
set of means by which one can express ideas to the computer.

Several experiments are described, aiming at finding synergies in the use of speech, gestures and sketching.

A sketching and speech application is presented, in which a user, besides the common shape recognition
of his sketches, has some of his speech reasoned into additional info. Talking about subject such as shapes,
spatial relations between them and their transformations in time rearranges the interpreted sketch
reflecting the acquired speech data.



\subsection*{Digital clay: deriving digital models from freehand sketches}

\begin{table}[htpb]
    \begin{tabular}{|r|l|}
        \hline
        tag & \cite{DIGCLAY00} \\
        authors & Eric Schweikardt and Mark D. Gross \\
        url & http://www.dtc.umn.edu/~ashesh/papers/gross-schweikardt-autoinconst.pdf \\
        year & 2000 \\
        cited by & 36 \\
        pages & 9 \\
        \hline
    \end{tabular}
\end{table}

%Digital Clay allows a user to draw freely and tries to convert the drawing into a 3D model.
%
%It uses two techniques to achieve that:
%
%\begin{itemize}
%	\item Huffman-Clowes algorithm, which identifies concave and convex vertices, requiring
%	every line to connect to another line and the program additionally demands the object
%	drawn to be solid;
%	\item infers 3D coordinates based on inherent rules that govern each type of drawing projection.
%	Examples: axes of isometric drawings have equal angles between them;
%	perspective drawings show foreshortening of lines as we get closer to the viewer
%\end{itemize}
%
%The downfalls of this method are the impossibility of describing occluded faces and the limited
%editing capacities available once the conversion has been done.



%\hrule
\newpage

\section*{City Modeling}



\subsection*{Continuous level-of-detail modeling of buildings in 3D city models}

\begin{table}[htpb]
    \begin{tabular}{|r|l|}
        \hline
        tag & \cite{LODCITY05} \\
        authors & J�rgen D�llner and Henrik Buchholz \\
        url & http://www.hpi.uni-potsdam.de/fileadmin/hpi/FG\_Doellner \\
        & /publication/2005\_ACMGIS\_doellner/2005\_Doellner\_ContinuousLODModeling.pdf \\
        year & 2005 \\
        cited by & 2 \\
        pages & 9 \\
        \hline
    \end{tabular}
\end{table}

%The project presents a solution for modeling buildings that feature a continuous level-of-detail.
%
%The authors propose the following levels of detail for a building (derived from CityGML):
%
%\begin{itemize}
%	\item simple block model
%	\item model with defined roof geometry
%	\item detailed indoor and outdoor building features
%\end{itemize}
%
%A building is composed of a list of floor objects. A floor object refers to a floor prototype, which
%contains the floor specification. This indirection allows a user to reuse floor prototypes in several
%floors of the same building and allows rendering optimizations.
%
%Each floor prototype is defined by its ground plan, which is one or more polygons that define the area
%on which walls can be constructed. There can be inner loops in order to allow features like courtyards.
%A ground plan supports thickness, useful for defining terraces.
%
%On top of a ground plan one can place walls. A wall represents a vertical, planar polygon. The default
%type of wall has no thickness, sufficient if a group of them forms a closed surface and can be only
%seen from the outside. Thick walls can also be added. A wall can be lower or higher than 
%its floor height, allowing balcony fronts and chimneys to be defined.
%
%The higher ground plans can have roofs. The most common roof types are supported (hipped, gabled, tent,
%mansard, pent, barrel) and a roof is described only by choosing the floor type and placing its
%most relevant points (know as the roof skeleton).
%
%Each floor prototype has a related floor decoration. A floor decoration is a collection of facade sections
%and window sections. The former allows whole wall sections to be assigned a material while the latter
%allows the definition of positioning and appearance of the floor's windows.

This article has an ordered and coherent solution for a building's hierarchy solution.
My project will benefit from this knowledge and I intend to apply a similar solution as long as it doesn't
constrain my users creation expressiveness.


%\subsection*{}
%
%\begin{table}[htpb]
%    \begin{tabular}{|r|l|}
%        \hline
%        tag & \cite{} \\
%        authors &  \\
%        url &  \\
%        year &  \\
%        cited by &  \\
%        pages &  \\
%        \hline
%    \end{tabular}
%\end{table}


