\section{Introdu��o}
A navega��o numa cena tridimensional processa-se normalmente usando um computador de secret�ria,
com rato e teclado como dispositivos de entrada e monitor como dispositivo de sa�da.
%De forma a tornar a navega��o no espa�o e actividades de transforma��o de objectos em experi�ncias imersivas
Foi desenvolvido no laborat�rio Prof. Louren�o Fernandes\cite{leme} um conjunto de modos de interac��o que se t�m provado eficazes para o efeito, recorrendo a equipamento de captura de movimentos e um ecr� de larga escala para navegar num
ambiente virtual e manipular objectos na perspectiva da primeira pessoa.

Este trabalho pretende proporcionar uma interac��o mais directa para navega��o tridimensional,
sem recurso a dispositivos intrusivos.
Recorre a comandos de voz para seleccionar a ac��o a executar e a movimentos dos bra�os para controlar o
desenrolar da mesma.
