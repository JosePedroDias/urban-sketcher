% Contexto + Problema

% http://en.wikipedia.org/wiki/Architecture
Architecture is the art and science of designing buildings and structures.
It is an interdisciplinary field and has similarities to applied science and
engineering, but unlike them, which focus more on functional and feasibility aspects of design, 
architecture deals with building costs, space and volume, texture and lighting
in order to achieve an aesthetically pleasing result.

For centuries methods and norms have arisen. Architects tools of work were based on paper and ruler.
With the evolution of computational power throughout the last century, 
increasingly more complete, fast and robust computer aided design (CAD) systems were developed and so did
devices capable of manipulating architectural entities -- such as the mouse, trackball, tablet, etc.

% only geographical, drawings?
Elaborating architectural designs usually starts by drawing rough sketches of the subject
in order to convey form, proportion and lighting. Relevant geographical data about the location
where the building is to be established is acquired. A series of two dimensional drawings is created
in order to precisely define building features and dimensions.
These drawings serve as input for civil engineers and the rest of staff responsible for constructing the building.
Three dimensional drawings and brochures are created and scale models are built, allowing
people with no architectural background to better perceive the building before it is built.

Nowadays architecture is slowly but steadily embracing usage of computers in the designing process.
Authoring systems such as Autodesk's AutoCAD
\footnote{AutoCAD -- http://www.autodesk.com/autocad}%\nocite{SITE-AUTOCAD}
are currently used to produce a set of views and floor plans necessary to construct a building.
These systems are optimized for such tasks, having limited support for general 3D volume and surface creation.

% more references to software?
The creation of 3D drawings and brochures for marketing purposes is common practice.
It relies in techniques such as
ray tracing\footnote{POVRay is a well known raytracer available at http://www.povray.org}%\nocite{SITE-POVRAY},
radiosity,
physically accurate light simulation
\footnote{Both these engines support realistic light simulation:
Maxwell -- http://www.maxwellrender.com%\nocite{SITE-MAXWELL};
Indigo -- http://www.indigorenderer.com%\nocite{SITE-INDIGO}}
}
High Dynamic Range Imaging (HDRI), etc.
These techniques produce believable results but take too much time to render in real time.

% frases enormes!
Another application of data coming from CAD systems is the generation of worlds optimized for navigation.
Performance is paramount in these systems so algorithms such as Binary Space Partition (BSP)
or octrees are applied for faster geometry rendering.
The compromise between believability and fast rendering times is assured with techniques such as
light maps and good usage of modern graphics card's Graphics Processing Unit (GPU),
allowing the application of many computer graphics algorithms that would otherwise be impossible
with such time constraints.
There are numerous 3D engines capable of presenting such worlds with good performance.
The major drawback in using them relies in how the user navigates the scene --
most systems use the popular interface featured in most first or third person shooters
available in the market,
relying on 4 keyboard keys for user translation plus user rotation with a mouse.
This interface works for youth game enthusiasts but presents an obstacle for most other people.
There are other navigation metaphors available such as flying but they keep using keyboard and mouse
as their input interface.

Exporting the geometry from a CAD program, which is a common solution, comes with several problems,
as stated by Alberto Raposo et al.\cite{CADVR06}:
\begin{itemize}
	\item \textbf{Low performance} -- due to unneeded model complexity;
	\item \textbf{Lack of realism} -- usually users of CAD programs don't associate material
	or texture data to objects;
	\item \textbf{Inadequate treatment of geometry} -- the conversion often occurs with loss of
	geometry, precision and errors such as badly oriented normals.
\end{itemize}

% client side benefits
Using a computer system to present a virtual building with the purpose of selling an idea,
doing real estate business or simulating virtual tours demands better metaphors,
better interface design and the least possible cluttering of the view in order
to achieve a richer user experience and get an overall good impression.

% architect benefits
From the architect's point of view, using such a system early in the process,
if preceded by geographical data capturing,
allows him to better perceive the construction area, which shall improve seamless building integration.
Additionally, applying such a system in the building design workflow allows the generation
of models able to complement or even replace the first prototyping stages,
commonly represented nowadays by rough sketches.

% document structure
The author starts by identifying the various domains in which problems arise.
Then a series of approaches to solve them is listed and the author reasons which ones better fit
the problem at hand.
Later on a comparative analysis between applications in the market allowing architecture modeling is conducted.
The document ends with a set of conclusions and points out directions for future work in
addressing this subject.
