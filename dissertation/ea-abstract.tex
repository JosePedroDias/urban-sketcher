%%% ABSTRACT %%%
\begin{abstract}


% problem

A system was developed for the creation of urban scenarios produced on large screen displays
using laser pointers as input devices and supporting collaborative usage.
A novel interface had to be developed to support interaction with such input devices,
based on the concept of gates and circular menus with a non-intrusive interface.

% implementation / features

Users can navigate on a virtual world using a set of comprehensive navigation modes, namely:
first person, bird's eye view and examine modes, along with a multimodal flight mode controlled by speech commands and arm tracking.
Simple shapes can be created and modeled using a minimalistic set of modeling tools, defining a novel modeling interface.
Buildings can be instantiated from a library of facade styles by drawing
the desired blueprint and setting the facade height to generate unique buildings.
Additional styles may be implemented by making use of a developed XML format for defining fa�ade layout rules.
Buildings and other shapes can be transformed and cloned.
Objects in the scene can have notes attached to them to aid in reviewing sessions.
An alternative building creation work flow is proposed to take advantage of this system for early prototypes
and showcasing projects.
%
%The system architecture is described throughly, followed by implementation details and evaluation tests.
%It is shown that users could successfully make use of the offered features based on a stroke-based
%interface and set of comprehensive menus. 
%The project conclusions and future work close this document.

\end{abstract}


\keywords{
	stroke,
	building,
	multimodal,
	large screen,
	BREP
}

