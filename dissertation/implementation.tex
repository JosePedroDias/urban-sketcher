\chapter{Implementation}

\TODO{intro to implementation}


\section{Navigation}

\subsection{First Person Mode}

This navigation mode is centered on the current point of view and works by triggering the displayed gates.
It features 8 gates: 4 direction displacement gates and 4 rotation gates.
To stop the movement/rotation one can either trigger the gate again, effectively disabling the ongoing action
or by triggering a different gate action.

The choice for this mode's layout suffered several evolutions.
At early stages opposing directions where placed at opposite sides of the ring but this
made correction by triggering the opposite action difficult, so opposing actions are now close together.
In this mode a restriction of one enabled action at a time is imposed to keep the handling easy for novice users.
%there as soon as an action starts the previous one stops.

There are gates for moving forward/backwards, up/down, pitch up/down and yaw up/down
\footnote{The verbs pitch and yaw come from flight semantics: to pitch is to look up/downwards; to yaw is to turn relatively to the ground plane.}.

\TODO{FIGURE}


\subsection{Compass Mode}

The compass mode was thought out for searching tasks. It allows the user to move along the ground plane and turn around it.

The compass navigation mode has 2 distinct areas:
the center circle displays a top-down view of the scene centered on the user;
the outer ring displays the main compass directions.

In order for one to move, a dragging movement must be performed inside the top-down view.
To reorient the user one must rotate the outer ring. The direction the user is facing can be read on the top part of the ring.

This mode could not be tested on multi-screen displays due to technical problems. It was enthusiastically accepted on
one-projector early tests.

\TODO{FIGURE}


\subsection{Examine Mode}

The examine mode allows the user to recenter attention on an object of the scene.
It features 3 gates and a center sphere.
The user must initially select the new center of attention by triggering the recenter gate,
finishing the stroke at the desired object/location.
Once the location is defined the remaining gates allow for zooming in and out the centered content
while the sphere allows for repositioning the user on the space around the object -- 
horizontal drags rotate horizontally, vertical moves vertically.

This mode is the most effective when performing shape modeling tasks.

\TODO{FIGURE}


\section{Content Creation}

The system's interface offers 3 families of shapes which can be instanced on the scene.
There are the primitives cube, cylinder and sphere; a set of previously generated shapes
and set of known building styles from which one can create buildings.
Primitives are the most versatile shapes since they support face and edge manipulation operations.
All shapes support simple transformations and cloning.

One uses building styles to create buildings on the scene. A library of generated shapes such as
people a trees serve as asserts to populate the scene with details and primitives can be instanced
as is or as building ground for custom shapes.

\TODO{FIGURE SHAPE MENU}


\subsection{Shape Instancing}

The instancing of shapes works using the Apply-to-Scene concept described on section \ref{design:apply-to-scene}.
Every gate of this type has a small arrow running outwards as a hint to the user of this feature.
The user activates the gate of the desired shape and ends the stroke where he wants it to rest.

\TODO{EXAMPLE IMAGE}


\subsection{Building Instancing}

%\TODO{ADD CALI TO BLOCKS, HERE AND TRIANGLE STROKE}
Once the building parameters have been gathered by the interface, as described on section \ref{design:building},
the building needs to be generated.
First the stroke is projected onto the construction plane and parsed by the shape recognizer as a rectangle.
The recognized rectangle dimensions serve as the blueprint which is extruded upwards for the measured height.
Then the building facades are generated according to the chosen style grammar and so is the ceiling.
The style grammar is fed each facade length and returns a set of spacings and facade elements that must be instanced
according to the rules defined in the style.
%If for instance the facade rule for a given floor says
%``center a door and use windows everywhere else'' the system measures the width of both door and window and returns
%the maximum number of windows and a door in the center that fit the facade.
To minimize the facade attachments in memory, a map of loaded attachments is managed so only the first instance of any
attachment is loaded.


\subsection{Building Style Grammar}

\subsubsection{Grammar Structure}

The building style grammar defines building parameters such as
\textbf{floor-height}, \textbf{ceiling} parameters and \textbf{color-interval}s for the walls and ceiling.
It also defines a set of rules for the generation of the facade attachments that make up the final facades,
defined by the optional \textbf{front-facade} and the \textbf{facades} elements.

One can define the layout of a floor with the \textbf{layout} element, composed of 4 sections:
\textbf{left}, \textbf{center}, \textbf{right} and \textbf{other}.
Of these only the \textbf{other} section is required and the layout works by trying to fill the facade space with \textbf{center}, \textbf{left} and \textbf{right}s' contents if those are present, repeating \textbf{other}'s contents for filling the remaining space.

Inside these sections one can put any of the \textbf{us-element}s: \textbf{atom}, \textbf{group}, \textbf{sequence} and \textbf{random}.
An \textbf{atom} is the simplest \textbf{us-element}, having the attributes
\emph{type}, \emph{spacing} and \emph{height}.
The \emph{type} parameter defines which shape to instantiate on the facade,
\emph{spacing} how many length of the facade it will consume and
\emph{height} can be used to shift the shape upwards (to move a window, for instance).

The remaining \textbf{us-element}s allow combining \textbf{us-element}s.
A set of \textbf{us-element}s inside a \textbf{group} create all content on the same place and measure the longest of its children.
A set of \textbf{us-element}s inside a \textbf{sequence} create all children one after the other.
The \emph{random} \textbf{us-element} is similar to \textbf{group}, but has the attribute \emph{odds},
a set of comma separated ratios defining the probability of each child to be picked.

Several floors can share the same layout. To apply a layout to one or a set of floors, the floor-span element exists.
It can have either the \emph{at} attribute defined or both \emph{min} and \emph{max},
resulting in the application of the enclosed layout to all the floors in the interval.

One can also define a different facade style for the front facade with the element \textbf{front-facade}.
This is useful when one wants to apply columns and doors to one facade but not the remaining ones.

\TODO{INSERT FIGURE WITH EXAMPLES OF ELEMENTS}

An example of a complete building style can be found on appendix \ref{residentialGrammar}.


\section{Content Editing}

All shapes in the system are made of 4-edged faces and all shapes are closed surfaces.
Operations such as the split by face loop rely on these properties.
Each shape computes each edge's neighboring faces (always 2)
and therefore each face's neighboring faces, forming an auxiliary structure called the edge map,
used for optimized queries for neighbors.

\TODO{summarize}

\subsection{Face Selection}

When a stroke finishes its start and ending points are projected onto the scene's available geometry.
If these 2 points lie on the same face of the same object that face is selected and the face contextual menu appears.

\TODO{Add face selection diagram and menu}


\subsection{Edge Selection}

If the stroke start and ending points lie on different but neighboring faces of the same shape, the edge between
those faces is selected and the edge contextual menu appears.

\TODO{Add edge selection diagram and menu}


\subsection{Determining and Selecting Directions}

In order to keep the interface simple and to minimize the number of steps to perform an operation, a set of directions
is estimated for edge and face selections -- these are believed to be the most frequently needed vectors for shape operations.
When an edge is selected the computed directions are the edge outwards normal and the directions from the edge along its neighboring faces.
When a face is selected the computed directions are the face normal along with the 4 directions from the center of the face
to each of its neighboring faces.

If an operation requires the user to select a direction from the user the computed directions are displayed centered on the selected aspect
and color coded. The interface relies on the user to keep drawing the stroke after the operation is triggered so the remaining of the stroke's direction serves to select the desired direction by approximation -- while the stroke is being drawn the operation is previewed by applying the currently most approximate vector direction to the shape. The length of the stroke can be also used from the provided information by using the stroke's direct length.

\TODO{EDGE DIRECTIONS, FACE DIRECTIONS}

\TODO{DIRECTION SELECTION}




\subsection{Shape Operations}

\TODO{OPERATIONS FIGURE}

By selecting a face the following operations can be performed:
\begin{itemize}
	\item \textbf{extrude face} - Extrude generates 4 additional faces, getting the direction from the selected face's normal and the displacement from stroke length.
	\item	\textbf{bevel face} - Bevel moves the face vertices, getting distance from stroke length.
	\item	\textbf{extrude \& bevel} - This is a compound operation of zero length extrude followed by bevel. It exists for convenience to aid in the creation of holes for features such as windows or chimneys.
	\item	\textbf{move face} - Move face displaces the face vertices, getting the direction from the selected face's normal and 4 computed directions.
	\item	\textbf{move coplanar faces} - This operation works as the previous one but the neighboring faces having an angle with the selected one smaller than the threshold are also affected by the movement. \TODO{EXAMPLE CYLINDER}
\end{itemize}

By selecting an edge the following operations can be performed:
\begin{itemize}
	\item \textbf{move edge} - Moves the edge vertices along one of the directions: edge normal and neighboring faces, obtaining the displacement from stroke length.
	\item	\textbf{split at face loop} - The crossed edge defines a direction. All faces on that face loop are split in the middle, creating additional geometry. This operation has instant application as it doesn't require additional data.
\end{itemize}

On both edge and face context menus there's an option to \textbf{toggle} the visibility of the \textbf{shape edges}.

All shapes have a stack of memento objects for supporting the undo operation, also available on both menus.
The memento design pattern states that an auxiliary object - the memento object - decorates the state of the key object structures so they
can be restored if backup is needed.


\section{Reviewing}

Notes can be created on the scene. A menu exists featuring a large area at the center where strokes can be drawn as if writing on paper.
A rubber option allows wiping the note contents. Once the note is written it can be attached to any scene object by using the
apply-to-scene concept. Notes are real 3D entities and can be hidden or edited by performing a closed stroke around them.

\TODO{CREATING NOTE, ATTACHED NOTE}


\section{Proposed Work Flow}

Shapes can be loaded or saved to XML and the simple descriptive format can be easily supported by 3D modelers.
For the purposes of the generation of terrains, library objects and facade attachments the Blender 3D modeling package
was used and an exporting plug-in developed.

\subsection{Scenario Creation}

The system is a prototype tool so not much emphasis was put into the importing of content.
Even so, different scenarios could be created.
The height data from a terrain patch can be represented by either:
\begin{itemize}
	\item an height map - a grayscale matrix with heighest values represented by brighter colors.
	\item a topographic map - a discrete set of contour lines uniting points with the same altitude.
\end{itemize}

With any of these data a 3D mesh can be generated - by vertex shifting a regular mesh for height maps or applying
a Delaunay triangulation to the contour lines.
With additional satellite imagery the terrain appearance could also be approximated by vertex coloring or texture projection application.


\subsection{Building Style Creation}

New attachment shapes could be created using Urban Sketcher itself or an auxiliary 3D modeler.
Creating a building style is a matter of editing the style grammar, defining ceiling type, colors, floor height and most of all
the layout for the building floors using existing and/or newly created attachments.


\TODO{UPDATED WORK FLOW DIAGRAM}

\TODO{example stages}

%\TODOL{IMMIVIEW OVERVIEW?}
%\input{dis-immiview}

%\section{General Interface}

\subsection{Strokes}

A \emph{stroke} is a continuous set of points drawn on our system.
Since the system works both on Tablet PCs and on power walls,
a stroke can be drawn either with the tablet's pen or with
a regular laser pointer.

Drawing strokes is therefore the main way of users interaction with
the system. It has been built to support several users interacting
simultaneously in the screen.

With this approach, any component in the system can't make use
of global mode information due to the fact that one can't know
for sure if the $ stroke_{n+1} $ was drawn by the same user than $ stroke_{n} $.

Therefore, strokes must serve many purposes, and in cases where a sequence
of actions must be done in order, they're achieved under the same stroke.

Another decision taken early on in the implementation was that there shall not be
fixed areas in the screen for contents such as menus. The initial state of the
system is completely free of interface widgets. When starting the system the users
see only a 3D perspective of the virtual world.


\subsubsection{Stroke Actions}

Any stroke can be interpreted as being one of (see figure \ref{fig:lasso-tri}):

\begin{description}
		\item[an open stroke] --
				any stroke where the segments don't intercept each other;

		\item[a lasso] --
				a closed stroke;

		\item[a triangle stroke] --
				this is a special case of a lasso, i.e. closed, where the stroke resembles a triangle.
\end{description}


\begin{figure}[!ht]
		\centering
		\includegraphics[width=3.5cm]{gfx/lasso.png}
		\qquad\qquad\qquad
		\includegraphics[width=3cm]{gfx/triangle.png}
		\caption{Lasso and triangle strokes}
		\label{fig:lasso-tri}
\end{figure}


\subsection{Gates}

As described earlier on in section \ref{sec:gate-concept} The Gate Concept,
a gate solves the problem of our input devices not being capable of detecting
click actions consistently.

\begin{figure}[!ht]
		\centering
		\includegraphics[width=5cm]{gfx/gate1e.png}
		\vspace{-0.5cm}
		\caption{Two gates: one featuring text and the other visual content}
		\label{fig:gate1}
\end{figure}

\begin{figure}[!ht]
		\vspace{-0.3cm}
		\centering
		\includegraphics[width=10cm]{gfx/gate2e.png}
		\vspace{-0.5cm}
		\caption{Gate activation}
		\label{fig:gate2}
\end{figure}

\begin{figure}[!ht]
		\vspace{-0.3cm}
		\centering
		\includegraphics[width=3cm]{gfx/gate3e.png}
		\vspace{-0.5cm}
		\caption{Gate showing its tooltip}
		\label{fig:gate3}
\end{figure}


Gates can either have text our visual content (fig \ref{fig:gate1}).
It was decided that gates shouldn't feature both at the same time.

The activation takes place by entering both the left side and the right
side areas of the gate consecutively (or the other way around).
The dashed line in figure \ref{fig:gate2} is only imaginary.

In order for users to inspect the meaning a gate with visual content, a
tooltip appears when the stroke is inside the gate,
which will happen prior to activation, so users can give it up (figure \ref{fig:gate3}).

In conclusion, for a user to activate a gate, an horizontal or quasi-horizontal
stroke must be drawn, as seen in figure \ref{fig:gate4}.

\begin{figure}[!ht]
		\centering
		\includegraphics[width=4cm]{gfx/activation.png}
		\caption{gate activation example}
		\label{fig:gate4}
\end{figure}

\subsection{Menus}

In order to get the main menu, the user has to draw a closed triangle stroke.
The menu appears at the center of the drawn stroke.

\begin{figure}[!ht]
		\centering
		\includegraphics[width=8cm]{gfx/menu.png}
		\caption{menu and its components}
		\label{fig:menu}
\end{figure}

Menus are round and have additional auxiliary gates and labels (figure \ref{fig:menu}).

In the bottom right area a label clearly identifies the functionality provided by the menu.

In the top right area menus feature a set of special gates.
The \emph{close gate} dismisses the menu.
The \emph{move gate}, when activated, makes the menu follow the stroke position until it ends.
The \emph{home gate} allows the user to go back to the previous menu, when that action applies.

Occasionally the menus are too complex and need to be segmented.
When that happens, mode selection gates appear to the left of the menu.

Different menus have different background colors.
They're translucent so the main perspective remains partially visible.
Menu colors were chosen to reflect the functionality, ex: all navigation menus are green.


%
\section{Navigation}

\subsection{1st Person Navigation Mode}

\TODO{ADD 1ST PERSON SHOT}
This widget has four moving directions and four rotations.
It allows to move forward/backward,
go up/down,
turn left/right and
pitch up/down.

Crossing a gate once starts the movement.
Crossing it again stops it.
If a gate is crossed while another one is active,
the first one stops -- this behavior was enforced
by user test inspection.
Having several moving/rotation
actions applied at the same this is confusing.
Clarity was favored over efficiency.

\subsection{Compass Navigation Mode}

\TODO{ADD COMPASS SHOT}

This is a direct application of the concept defined
in section \ref{sec:bev} Bird's Eye View Mode, on page \pageref{sec:bev}.

The outer ring of the widget allows the user to see and change
the current orientation in terms of cardial points positioning.

The inner circle allows inspection of the top-down view, with the user
being in the center. Applying strokes inside this area drags the map,
changing user's position in the ground plane.

\subsection{Examine Navigation Mode}

\TODO{ADD EXAMINE SHOT}

\TODO{ADD VIRTUAL SPHERE ILLUSTRATION}

This widget features a wire frame sphere at the center, two zooming gates
and two additional gates whole purpose is described below.
It is a direct application of the concept defined
in section \ref{sec:examine} Examine Navigation Mode, on page \pageref{sec:examine}.

The sphere occupying the center of the widget simulates the virtual sphere centered
on the subject of interest with radius $ dist $.

The user is at distance $ dist $ from the subject of interest.
This distance can be increased or decreased by activating the zoom gates.
Movements in the $ XX $ direction are translated into rotations around $ \theta $, while
movements in the $ YY $ direction translate into rotations around $ \phi $.

Two additional gates allow changing the subject of interest.
They work by crossing and ending the stroke at the target.
The first one, \emph{point of interest center}, allows one to examine the center of the object.
The remaining one, \emph{point of interest spot}, changes the examined position to be the surface
position where the stroke ``touches''.
The former is useful for inspection of the overall object while the latter is of great
use in editing or inspecting surface details.

\subsection{Fly Navigation Mode}

%intro DONE

\begin{figure}[!ht]
		\centering
		\includegraphics[width=6cm]{gfx/don.png}
		\caption{user navigating with the multimodal navigation mode}
		\label{fig:don}
\end{figure}

Navigation in a 3D scene takes place normally using a desktop computer
with the mouse and keyboard as input devices and the monitor as output device.
In the multimedia lab there was a motion tracking system available.
It was put to use in conjunction with a wireless microphone to allow for an
alternative mode of navigation. A set of reflective markers is attached to
the user's arms and the wireless microphone attached to his/her ear.
The microphone allows for enabling and disabling the multimodal navigation mode
by issuing respectively the commands \textbf{begin flying} / \textbf{stop flying}.

\subsubsection{Physical Configuration}

Motion tracking took place using reflective markers attached to the user's arms (figure \ref{fig:markers-cams} left).
A set of 4 cameras with infrared lights were installed in the lab, which each
captured image being processed by a computer identifying the visible markers in
image-space, making the computation of the spacial position of each marker by
merging the data from the 4 captured views and labeling each tracked marker
from the initial extended arm pose (figure \ref{fig:markers-cams} right).

The Microsoft Speech API 5.1, American English version was used to recognize the
voice commands. Capturing of the commands was done by attaching a wireless microphone
to the user's ear.

%ap�s a compara��o do reconhecimento do mesmo face a um microfone fixo e um Bluetooth.

%O reconhecedor mostrou-se algo limitado, tendo sido necess�rio adaptar a gram�tica a palavras n�o amb�guas entre si ou
%que gerassem falsos positivos.
%
%%Preliminarmente realizou-se um teste para escolher o melhor hardware para reconhecimento de voz, entre um microfone Wireless, %um fixo a captar toda a sala e um auricular Bluetooth. Os comandos dispon�veis na aplica��o foram ditados repetidas vezes por 3 %utilizadores, em ordem aleat�ria e com os 3 dispositivos.
%
%%Conclui-se que o microfone Wireless era a melhor op��o com uma taxa de reconhecimento m�dia de 99\%, mas para dist�ncias curtas %(\(<\)3m) o auricular Bluetooth teve resultados semelhantes, com 98\%. O microfone fixo captou muito ru�do e produziu resultados %muito insatisfat�rios de menos de 50\%.

%O reconhecedor mostrou-se algo limitado, tendo sido necess�rio adaptar a gram�tica a palavras n�o amb�guas entre si ou
%que gerassem falsos positivos.

\begin{figure}[!ht]
		\centering
		\includegraphics[width=7cm]{gfx/markers2.png}
		\qquad
		\includegraphics[width=5cm]{gfx/4cam-setup.png}
		\caption{markers and camera setup}
		\label{fig:markers-cams}
\end{figure}
%
%
%\begin{figure}[!ht]
%		\centering
%		\includegraphics[width=6cm]{gfx/markers2.png}
%		\caption{markers and microphone for tracked user}
%		\label{fig:markers}
%\end{figure}
%
%\begin{figure}[!ht]
%		\centering
%		\includegraphics[width=6cm]{gfx/4cam-setup.png}
%		\caption{four cameras setup for motion tracking}
%		\label{fig:cam-setup}
%\end{figure}


\subsubsection{Interaction}

The multimodal navigation mode requires the user to have his / her arms extended towards the screen.

Controlling the flight speed works by the measurement of the distance between hands:
the closer they are from each other, the faster the flight speed is.
If the arms get close to a ninety degrees angle between them the flight halts (figure \ref{fig:fly-speed}).

Changing the flight orientation relatively to the ground plane is achieved by setting the arms
angle with the ground at opposing directions, with a bigger difference between angles generating
a faster rotation movement. If the user wants to turn right, for instance, he / she has to
raise the left arm and lower the right one (figure \ref{fig:fly-rot-up}, left).

To change flight altitude, both arms must be oriented in the same direction relatively to the ground plane,
either both raised or both lowered.
Again, the higher the angle is from the original arms extended forward pose,
the bigger the flight altitude shift occurs (figure \ref{fig:fly-rot-up}, right).

\begin{figure}[!ht]
		\centering
		\includegraphics[width=5cm]{gfx/immi-fly-fast.jpg}
		\qquad
		\includegraphics[width=5cm]{gfx/immi-fly-slow.jpg}
		\caption{control flight speed}
		\label{fig:fly-speed}
\end{figure}

\begin{figure}[!ht]
		\centering
		\includegraphics[width=5cm]{gfx/immi-fly-rotate.jpg}
		\qquad
		\includegraphics[width=5cm]{gfx/immi-fly-up.jpg}
		\caption{rotate flight orientation and increasing flight altitude}
		\label{fig:fly-rot-up}
\end{figure}

%
%\TODO{IMMI PAPER}


%
\section{Building Creation}

This is the application of the concepts of Building, Architecture Style and
Facade Attachment defined in section \ref{sec:concepts} Concepts on page \pageref{sec:concepts}.

\subsection{Task Definition}

In order to create a building, the user has to position himself close to the terrain area
where the building is destined to be.
He must then summon a construction plane by invoking the main menu (closed triangle stroke),
choose the create option and do a drop stroke, activating the construction plane gate and
continuing the stroke until the desired position is met.

\TODO{ADD STROKE EXAMPLE}

Once the plane is displayed, the user has to draw a stroke resembling a closed rectangle
-- this rectangle defines the blueprint of the facades --
and keep drawing upwards to define the building's height.

A menu appears then, asking the user for the purpose of this building.
The displayed options are the set of available architecture styles.
The user chooses one and the building's walls are colored and textured,
with facade attachments populating the facade walls and a roof added,
all according to the chosen architecture style.


\subsection{Creation of Facade Attachments}

A facade attachment can be accomplished by modeling an object in Blender,
setting the mesh name to the desired name and applying the export to urban sketcher plug-in.
This generates a shape with the format of custom shapes, as described in section \ref{sec:shape-persistence}
on page \pageref{sec:shape-persistence}.
The modeler units are interpreted as meters and the object's origin position should be the facade's floor,
which makes windows have a low origin position, for example.
\TODO{SCREENSHOT?}


\subsection{Creation of an Architecture Style}

\TODO{DESCRIBE GRAMMAR}

\TODO{EXPLAIN POPULATING ALGORITHM}


%\section{ImmiView Framework}

\subsection{Architecture}


%
\section{Custom Scenery Creation Work Flow}

\TODO{blender, XML...}

\TODO{summary of implementation}


