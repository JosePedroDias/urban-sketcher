% contexto
Architects need to draft sketches of building designs prior to drawing
the final CAD documents in order to study viable designs and discuss them with clients.
This is currently done with pen and paper. One could offer architects an alternative
method to draw their sketches, one that models the buildings' drafts in 3D, allowing
editing shapes, navigating the scene and manipulating additional content,
such as annotations.
%This solution would aid clients in perceiving the architect's mindset, 

The author is committed to develop such a system using currently available technology
and multimodal interfaces.
%a system with the purpose of creating
%three dimensional drafts of architecture designs using currently available technology
%and multimodal interfaces in order to fill this need.

%In building such a system one faces several problems that need to be solved.
The author identifies several problems and lists possible solutions for them.
A comparative analysis of available systems for architecture creation is then conducted.
Finally, conclusions are reached and directions for future work in this area defined.

% problemas
%Building such a system involves solving the following problems:
%one must find a useful representation for the urban scenery,
%offer a proper input interface for the users,
%allowing a simple way of sketching
%geometry with an easy learning curve, helpful user aids via suggestions and a fluid and efficient
%navigation interface.

% solu��es existentes
%Several solutions for the problems stated above are reviewed.

% compara��o
%A comparative analysis is performed between available products available in the market
%which architects use for building design.

% conclus�o
%The document ends with a list of conclusions related to the above topics.


\textbf{Keywords:} urban, city, sketch, model, architecture, building

%\hrule