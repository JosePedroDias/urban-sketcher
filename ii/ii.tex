\documentclass[a4paper,10pt]{article}
\usepackage{geometry}          		 % allows changing page dimensions
\geometry{a4paper,left=2.5cm,right=2cm,top=2.5cm,bottom=1.5cm}
\usepackage[latin1]{inputenc}  		 % supports the pt encoding
\usepackage{graphicx}         		  % support graphic files
\usepackage{color}            		  % should allow colored text
%\usepackage{url}            		    % should identify and format urls
%\usepackage{natbib}             		% pretty bibtex
%\usepackage{hyperref}							% good links for acrobat, colors, etc.
%\hypersetup{final,colorlinks=true}	% sets up hyperref
\pagestyle{empty}										% removes page numbering

\begin{document}
    \begin{center}
        \begin{Huge}
            \textbf{creation of urban scenery\\using multimodal interfaces}
        \end{Huge}
        \begin{Large}
            \\[\baselineskip]
            \textbf{-- Introdu��o � Investiga��o --}
            \\[\baselineskip]
            Jos� Pedro Dias, n.48296
            \\
            IST Tagus Park
        \end{Large}
    \end{center}
    \hrulefill
    
    \baselineskip = 18pt	% bigger space between lines
    
    My thesis aims at developing a program to allow the creation of buildings in urban sceneries making use of multimodal
    interfaces. My project was made part of the Improve\footnote{Read the first entry of my blog,
    http://urbansketcher.blogspot.com for an extensive explanation of Improve and the whole project.} consortium
    - a group of european universities and companies,
    the former providing developers and scientific innovation;
    the latter providing experience in the fields of architecture and the automotive industry.
    Being part of Improve, the project inherits requirements both in terms of functionalities to provide,
    hardware to use and a set of frameworks to work on.
    
    My program is intended to be run in a collaborative scenario where both people using tablet PCs and using a laser
    pointer in front of the powerwall\footnote{The powerwall consists of a cluster of 4 x 3 projectors and their
    controlling computers, set up in Louren�o Fernandes lab at IST Tagus Park.} can generate and manipulate buildings,
    navigate the city and read or post annotations on the geometry of the world.
    
    This project must therefore provide innovative solutions for the inputting of geometry, navigation in the scene and
    the interface itself, since the input hardware for the users is expected to be the pen (on the tablet PC scenario)
    or a laser pointer on the powerwall (on the powerwall scenario). Bear in mind that these devices, though somewhat
    similar to the mouse device, have different requirements: there are no buttons available, have worse precision, etc.
    
    I will do an extensive evaluation of the current state of the art software in terms of
    geometry modelling with a simple interface - Google SketchUp\footnote{Google Sketchup is a free software available
    at http://www.sketchup.com/} - in order to identify good solutions in use by the software and
    aspects that could be improved.
    
    I have divided my article research into the following areas:     
    \begin{itemize}
        \item \textbf{City Building}
            -- ways of inputting building data, edit it and represent it.
        \item \textbf{Sketch Recognition}
            -- algorithms and techniques for interpreting sketches.
        \item \textbf{Stroke Recognition}
            -- algorithms for stroke recognition. This is lower level when compared to sketch recognition.
    \end{itemize}
\end{document}
