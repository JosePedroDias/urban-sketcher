\abstract{
Navega��o em cenas tridimensionais em ambientes de visualiza��o de larga
escala constitui um desafio para a realiza��o de interfaces simples, mas
expressivas e f�ceis de aprender. Este artigo apresenta uma solu��o
baseada em interfaces multimodais, combinando captura de movimento,
gestos e fala que permite a utilizadores navegarem na primeira pessoa,
com as m�os livres num mundo virtual representado num ecr�
multi-projec��o de larga escala. Para tal utilizamos c�maras de
infravermelhos que seguem marcadores fixados ao corpo dos utilizadores.
O sistema utiliza a met�fora do Super-Homem para permitir voo controlado
utilizando movimentos dos bra�os. De modo semelhante � poss�vel
manipular e transformar objectos. Este documento descreve os princ�pios
e alguns detalhes da nossa abordagem, descrevendo testes efectuados com
utilizadores e uma discuss�o dos resultados.
%Aqui se apresenta o trabalho desenvolvido na procura de um modo alternativo de navega��o e manipula��o de objectos
%frente a um sistema multi-projec��o, onde � projectado um mundo virtual na perspectiva do utilizador.
%O mesmo deve equipar-se com marcadores reflectores nos bra�os, sendo o seu movimento capturado por 
%um sistema de captura de movimentos capaz de determinar com precis�o a localiza��o de cada marcador no espa�o em frente �tela.
%Mapeou-se a interac��o de voar sobre a cena erguendo os bra�os para mimificar o voo do Super-Homem. Met�foras semelhantes
%foram utilizadas para imprimir transforma��es em objectos.
%Explicar-se-� neste documento o modo de funcionamento deste prot�tipo, problemas encontrados assim como testes efectuados
%com utilizadores, e uma an�lise dos que foi conseguido.
}

\keywords{captura de movimentos, manipula��o directa, navega��o, manipula��o de objectos, interac��o por via de gestos.}

