\section{Representa��o Interna}

A estrutura do modelo geom�trico proposta � baseada em faces de 4 v�rtices. 
Esta estrutura implica que:
cada aresta seja partilhada por duas faces;
cada aresta pertencente a uma face tenha uma face oposta.

Al�m de guardar a lista de v�rtices e respectivas posi��es, assim como a lista de faces
(uma face � uma lista de quatro �ndices de v�rtices), � gerada uma estrutura auxiliar,
o mapa de arestas, que associa a cada aresta as faces que lhe s�o vizinhas. Esta informa��o
vai permitir o c�lculo de diversos vectores auxiliares assim como tornar poss�vel a opera��o
de corte de \textit{loop} de faces.

%Outras estruturas s�o igualmente definidas, como um lista de cores ou materiais por face.

O mecanismo de \textit{undo} foi implementado, baseado no padr�o de desenho Memento \cite{despat}.
Cada objecto tem associada uma pilha de estados de modo a poder regressar a qualquer passo anterior de modela��o.

A persist�ncia de objectos recorre a uma estrutura simples em XML, como se ilustra:

%\textbf{Salvaguarda e Carregamento} --
%uma forma geom�trica pode ser guardada para posterior carregamento num formato
%XML definido. O mesmo segue o seguinte padr�o:
\begin{scriptsize}
	\begin{verbatim}
	<?xml version="1.0" encoding="utf-8"?>
	  <shape>
	    <vertices count="56">
	      <vertex x="-1" y="-0.627882" z="5.29942"/>
	      ...
	    </vertices>
	  <faces count="54">
	    <face v1="0" v2="1" v3="2" v4="3"/>
	    ...
	  </faces>
	  <colors count="54">
	    <color r="0.9" g="0.9" b="0.9"/>
	    ...
	  </colors>
	</shape>
	\end{verbatim}
\end{scriptsize}