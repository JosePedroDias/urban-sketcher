\chapter*{Resumo}
\addcontentsline{toc}{chapter}{Resumo}

%\TODO{CHECK FOR MAXIMUM 250 WORDS!}

Desenvolveu-se um sistema para ecr�s de larga escala,
fazendo uso de ponteiros laser convencionais como interface de entrada.
Este sistema suporta uso colaborativo.
Foi criada uma interface inovadora de modo a suportar a interac��o com tais dispositivos,
baseada no conceito de port�es e menus circulares n�o-intrusivos.



% implementation / features


Os utilizadores s�o capacitados de navegar num mundo virtual utilizando um conjunto de
modos de navega��o, nomeadamente: primeira pessoa, modo b�ssola e modo examinar, assim como
um modo de v�o multimodal, integrando comandos de voz e movimento dos bra�os.
Podem ser criadas e modeladas formas simples, fazendo uso de um conjunto de ferramentas
de modela��o que definem um novo modelo de interac��o para a modela��o de objectos.
Podem ser instanciados edif�cios a partir de um conjunto de estilos de fachada pr�-existentes,
pelo desenho da sua planta e al�ado, dando origem a edif�cios �nicos.
Estilos adicionais podem ser gerados, recorrendo a um formato XML para definir regras de distribui��o de elementos nas fachadas de edif�cios.
Os edif�cios e restantes objectos podem ser transformados e clonados.
Podem anexar-se anota��es a objectos, de modo a suportar cen�rios de revis�o.
Um processo alternativo de cria��o de edif�cios � proposto para tirar proveito
do sistema tanto nas primeiras fases como na apresenta��o de projectos.

A arquitectura do sistema � descrita, seguida de detalhes de implementa��o e testes de avalia��o.
� demonstrado que os utilizadores conseguiram fazer uso das capacidades do sistema com sucesso.
As conclus�es do projecto e trabalho futuro fecham este documento.


%\TODO{CHECK FOR 4 -- 6 KEYWORDS!}
\textbf{Palavras-Chave:}
tra�o,
edif�cio,
interface multimodal,
navega��o,
ecr� larga escala,
BREP
