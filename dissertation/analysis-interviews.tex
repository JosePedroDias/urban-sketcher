\section{Interviews with Architects}

Two interviews were conducted with experts in this domain.
The interview with Architect Jos� Seco focused mainly on ways to classify
buildings while the interview with Prof. Lu�sa Caldas focused on
understanding the current architectural work flow and ways to improve it.


\subsection{Building Characteristics}

A building can be described in several dimensions.
Its purpose: whether it is a residential building, office, retail commerce, factory facilities, schools, hospitals, leisure, etc.
Its height: defined by the number of allowed floors which comes from a maximum floor height,
defined by the local authorities based on the population density defined for the construction area.
Blueprint Configuration: influenced by the topography, terrain type, access routes (roads, etc).
A building may also have different floors serving different purposes -- for instance
the first floor having small commerce and the remaining ones for residential purposes.
Another legal subject is the useful construction area ratio, defining the terrain to leave unbuilt, green areas, etc.
The ceiling of buildings can be classified by either being terrace-based (plain) or slanted.

There are three common relative distribution of buildings:
\begin{itemize}
	\item individual buildings, with a large area separating each one;
	\item twin facades, with one wall shared by each pair of consecutive buildings;
	\item lined buildings, with a set of buildings lined up in a direction, with a minimum distance between them.
\end{itemize}

%\TODOL{FIGURE SHOWING RELATIVE DIST OF BUILDINGS}

There are additional issues regarding building planning.
Most materials should ideally come from nearby natural resources;
the glassed areas in the facades respect a ratio (ex: 10\% of the facade).


\subsection{Work Flow Improvements}

Nowadays most town halls can provide digitally accurate data of the area the architectural
project is intended to occupy.
The most common and useful documents are aerial photographies of the terrain
and topographic maps,
providing discrete but accurate measuring of the area topography.
Making use of provided data early on in the project would improve the work flow dramatically.
Most architects and engineers need to set the exact place where the project is to be developed
by referencing contour line maps or effectively scouting the area on foot.

An alternative solution would be to feed the system with this data and get a simulation of the actual 3D
area, both in terms of appearance and volume. If the system could simulate lighting conditions
-- i.e. Sun at different day times and different months
-- lighting studies could be conducted with early sketch studies of the main volumes tried at different locations.
A good cross-section tool would be of great use for this task too, allowing a more effective evaluation
of the interaction of the building and its surrounding terrain at different angles.

The development of the main building shapes is an iterative process with strong subjective aspects.
It can begin with box shapes making up the volumes with different lighting conditions being applied to them.
From then an extrusion and oversketching process would shape up the definite edges and curves.

%\TODOL{FIGURE SECTION, DIFFERENT LIGHTING CONDITIONS, REDEFINITION OF VOLUMES}

If on-site visits to the terrain are made, pictures and video footage could be captured and its contents
attached to their virtual counterparts, so that by exploring the virtual representation of the construction
area one could make use of real views captured earlier.
This would allow other members of the team to get a more integrated presentation of this data,
offering a straightforward way of later referencing them by overlooking the virtual scene.

\subsection{Discussion}

From the various discussed building properties, some might be set by the user
while others could be derived from related data.
Strong candidates for properties set by user intervention are both the
buildings blueprints and height.
The placement of the buildings should be free but means should be given so that
making blocks of similar buildings would become a less complicated process.
The system should provide fast creation of building facades.
Architectural styles should map the most common building purposes and styles so that the creation
of a building to be a matter of defining its placement, volume and purpose.
The materials, floor heights and roof types of buildings should also be mapped into architectural style properties.

Regarding the creation of the surrounding environment where the project is to take place,
a more tight integration between provided content and the projected architecture could be accomplished.
Even if not directly supported by the system, the usage of height maps / contour maps and aerial photography
for the generation of the virtual scenario where the building creation and experimenting is to take place should
be streamlined with a work flow.
Nowadays the software used by architects does not allow them to integrate such data, let alone the correct exploitation of its potential.
Therefore, a well thought out navigation system should be provided so that a creative user could easily find areas for
construction and client users could explore the virtual scenario.

