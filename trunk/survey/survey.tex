%\documentclass[a4paper]{llncs}
\documentclass{llncs}
\usepackage{llncsdoc}
\usepackage[latin1]{inputenc}   % support the pt encoding
\usepackage{graphicx}   		    % support graphic files
\usepackage{subfigure}   		    % for subfigure usage
\usepackage{enumerate}					% allows enumerations a)... I..., 1... etc.

%\usepackage{url}
%\usepackage[pdftex,
%	pdftitle					= {creating urban scenery using multimodal interfaces},
%	pdfauthor					= {Jos� Pedro Dias},
%	bookmarks         = true,				% do bookmarks
%	bookmarksnumbered = true,				% keep sedtion numbers
%%	pdfpagemode       = None,				% start with closed bookmarks
%	pdfstartview      = FitH,				% starting view scale
%%	pdfpagelayout     = SinglePage,	% one page at a time
%	linkcolor					= black,			% links inside the document
%	citecolor					= black,			% for cites
%%	linkcolor					= blue,				% links inside the document
%%	citecolor					= red,				% for cites
%%	urlcolor					= green,			% for urls
%	colorlinks        = true]{hyperref}

\usepackage{pict2e}							% these 2 are for table a\b cells
\usepackage{slashbox}						% pict2e is optional but enhances the result

%\def\hellow{
%	\textcolor{red}{HELLO!}
%}
%
%% with receiving param
%\def\hellow2#1{
%	\textcolor{red}{begin #1 end}
%}
%
%% alternative way of starting ending a command
%\def\hellow3#1{
%	{\bf#1}
%	\begin{bf}#1\end{bf}
%}


% show TODOs
\ifpdf
	\def\TODO#1{
		\textcolor{red}{#1}
		%\addcontentsline{toc}{subsection}{To Do}
	}
\else
	\def\TODO#1{
		\textbf{TODO:} #1
	}
\fi

% hide TODOs
%\def\TODO#1{}


% grades!
\def\GdE{\includegraphics[width=0.35cm]{gfx/grade1.png}}
\def\GdD{\includegraphics[width=0.35cm]{gfx/grade2.png}}
\def\GdC{\includegraphics[width=0.35cm]{gfx/grade3.png}}
\def\GdB{\includegraphics[width=0.35cm]{gfx/grade4.png}}
\def\GdA{\includegraphics[width=0.35cm]{gfx/grade5.png}}
									% \TODO
% LANGUAGE PROBLEMS:
% too much the's and a's
% too long sentences, use commas
% write in the third person singular

% NEW WORDS:
% thus, ent�o
% among, de entre (os)

% FROM BRUNO
% one, the author -> demasiados!
% abstract n est� a resumir este doc
% falta mat�ria de IMMI (refs multimodal!. voz, hapticas, tracking)
% maneiras f�ceis de adquirir 3d scanning, sat�lite
% interfaces de navega��o -> google earth, outros!
% lod / clustering...
% o que podemos construir com cada solu��o
% t�cnicas de visualiza��o (luz, sombra, npr, etc.) piranesi 
% para cada solu��o enunciada explicar q problema resolve
% tabela na sec��o 3
% opensg, openscenegraph urls
% componente colaborativa
% dassault systeme (3ds.com) corporate ad campaign

% ARTICLES TO FETCH/READ:
% tiago g.
% lemewall, outros sistemas relacionados
% + sistemas de sketching (+ desenhar no espa�o)
% parte 'the problem' com os problemas da abordagem actual
% crossy, flowmenu, etc.
% speech recognition, motion tracking, etc.
% 


\begin{document}

\title{creating urban scenery\\using multimodal interfaces}
\author{Jos� Pedro Dias}
\institute{IST,Universidade T�cnica de Lisboa\\Rua Alves Redol, 1000-029 Lisboa\\
\email{jose.pedro.dias@gmail.com}}

\maketitle

\begin{abstract}
% contexto
Architects need to draft sketches of building designs prior to drawing
the final CAD documents in order to study viable designs and discuss them with clients.
This is currently done with pen and paper. One could offer architects an alternative
method to draw their sketches, one that models the buildings' drafts in 3D, allowing
editing shapes, navigating the scene and manipulating additional content,
such as annotations.
%This solution would aid clients in perceiving the architect's mindset, 

The author is committed to develop such a system using currently available technology
and multimodal interfaces.
%a system with the purpose of creating
%three dimensional drafts of architecture designs using currently available technology
%and multimodal interfaces in order to fill this need.

%In building such a system one faces several problems that need to be solved.
The author identifies several problems and lists possible solutions for them.
A comparative analysis of available systems for architecture creation is then conducted.
Finally, conclusions are reached and directions for future work in this area defined.

% problemas
%Building such a system involves solving the following problems:
%one must find a useful representation for the urban scenery,
%offer a proper input interface for the users,
%allowing a simple way of sketching
%geometry with an easy learning curve, helpful user aids via suggestions and a fluid and efficient
%navigation interface.

% solu��es existentes
%Several solutions for the problems stated above are reviewed.

% compara��o
%A comparative analysis is performed between available products available in the market
%which architects use for building design.

% conclus�o
%The document ends with a list of conclusions related to the above topics.


\textbf{Keywords:} urban, city, sketch, model, architecture, building

%\hrule
\end{abstract}

\section{Introduction}
% Context + Problem

% http://en.wikipedia.org/wiki/Architecture

\subsection{The Evolution of Architecture}
Architecture is the art and science of designing buildings and structures.
It is an interdisciplinary field which has similarities to applied science and
engineering, but unlike them, focusing on functional and feasibility aspects of design, 
architecture deals with building costs, space and volume, materials and lighting
in order to achieve an aesthetically pleasing result.

For centuries methods and norms have arisen. Architecture tools of work were based on paper and ruler.
With the evolution of computational power throughout the last century, 
increasingly more complete, fast and robust Computer Aided Design (CAD) systems were developed and so did
devices capable of manipulating architectural entities -- such as the mouse, trackball, tablet, etc.

\subsection{Current Workflow for Building Design}
Elaborating architectural designs usually starts by drawing rough sketches of the subject
in order to convey form, proportion and lighting.
Relevant geographical data about the location where the building is to be established is acquired.
A series of two dimensional drawings is created in order to precisely define building features, dimensions and location.
These drawings serve as input for civil engineers and the rest of staff responsible for constructing the building.
Three dimensional drawings and brochures are created and scale models are built,
allowing people with no architectural background to better perceive the building before it is built.

Nowadays architecture is slowly but steadily embracing usage of computers in the designing process.
Authoring systems such as Autodesk's AutoCAD
\cite{SITE-AUTOCAD}
are currently used to produce a set of views and floor plans necessary to construct a building.
These systems are optimized for such tasks, having limited support for general 3D volume and surface creation
and rely heavily in desktop interfaces with keyboard/mouse.

The creation of 3D drawings and brochures for marketing purposes is common practice.
It relies in techniques such as
ray tracing \cite{SITE-POVRAY},
radiosity,
physically accurate light simulation \cite{SITE-MAXWELL}, \cite{SITE-INDIGO}; 
High Dynamic Range Imaging (HDRI), etc.
These techniques produce believable results but take too much time to render in real time.

% articles for bsps, octtrees, light rendering alg?
Another application for data coming from CAD systems is the generation of worlds optimized for navigation.
Performance is paramount in these systems so algorithms such space partition --
Binary Space Partitions (BSPs) for closed space rendering,
OctTrees for open space rendering --
and multiple Levels Of Detail (LOD) for each shape are commonly used.
The compromise between believability and fast rendering times is assured with techniques such as
light maps and using recent graphics card's Graphics Processing Unit (GPU),
shifting complex computer graphics algorithms out of the Central Processing Unit (CPU),
obtaining frame rates otherwise impossible with current hardware.

There are numerous 3D engines capable of presenting such worlds with good performance.
The major drawbacks in their usage rely on how the user navigates the scene --
most systems use the popular interface featured in most first or third person shooters,
relying on mouse/keyboard for input --
and on the lack of support for out-of-the-box interaction with shapes.
\TODO{UGLY PHRASE!}

\newpage
Exporting the geometry from a CAD program, which is a common solution, comes with several problems,
as stated by Alberto Raposo et al.\cite{CADVR06}:
\begin{description}
	\item[Low performance] -- due to unneeded model complexity;
	\item[Lack of realism] -- usually users of CAD programs don't associate material
	or texture data to objects;
	\item[Inadequate treatment of geometry] -- the conversion often occurs with loss of
	geometry, precision and errors such as badly oriented normals.
\end{description}

\subsection{Benefits and Goals of an Integrated Approach}
% client side benefits
Using a computer system to present a virtual building with the purpose of selling an idea,
doing real estate business or simulating virtual tours demands better metaphors,
a better interface design and a gentle learning curve to the user
achieves a richer user experience and gets an overall better perception of the project.
Navigating and reviewing content should be simple tasks to perform.

% architect benefits
From the architect's point of view, using such a system early in the process,
if preceded by geographical data capturing,
allows a better notion of the construction area, which improves a seamless building integration.
Additionally, applying such a system in the building design workflow allows
the generation of models able to complement or even replace the first prototyping stages,
commonly represented nowadays by rough sketches.
The possibility of collaborative remote design and content reviewing offers a cheaper
alternative for distributed projects and allows a shorter validation cycle with the clients.

\subsection{Document Structure}
% structure for the rest of the document
This document continues with
the analysis of a series of projects in this area, followed by.
the identification of various issues that need to be solved in order to fulfill these goals:
input modalities (using laser pointers, tablet PC pen, motion tracking and voice commands);
output modalities (using tablet PCs, powerwalls or head mounted displays);
shape creation and transformation (for modeling simple buildings, their translation, scaling, etc.);
scene navigation (using several modalities) and
content reviewing (by creating and editing shape attached annotations).
Later on a comparative analysis between applications in the market allowing architecture modeling is conducted
and the document ends with a set of conclusions and directions for future work in addressing this subject.


\section{The Problem}
\TODO{still lacks better intros}
% problem intro
Currently an architect starts the building design work flow by drawing rough
free form sketches of the desired building. These drawings are highly subjective
and clients have a hard time in their interpretation and validation.
The information captured in such drawings has little use in the remaining stages
of building design.

\TODO{insert draft sketches}

There are systems which are commonly used for professional building layouts
such as Autodesk Autocad \cite{SITE-AUTOCAD}.
The architect replicates his former ideas in the rigid standardized interface
offered by these systems. This is a rigorous endeavor, taking much time to be asserted.

After the main project documents are issued comes the time when the idea has to be pitched to end clients.
In order to gather buyers and investors its common to generate colorful previews of the building featuring increasingly realistic features such as detailed materials, light propagation and crowd simulations.
See Fig. \ref{FIG-REALISTIC}.

\begin{figure}[!ht]
	\centering
	\includegraphics[width=6cm]{gfx/realistic01.jpg}
	\includegraphics[width=6cm]{gfx/realistic03.jpg}
	\caption{Realistic renderings done with Maxwell Renderer}
	\label{FIG-REALISTIC}
\end{figure}

% 3 stage process
\TODO{Redo workflow figures}
Despite the ongoing advances in computation, the architectural workflow keeps these three separated stages
that don't share media or applications (Fig. \ref{FIG-WORKFLOW1}).
Both the conceptual design stage and the final reviewing and marketing stage would benefit from 
an integrated system with a comprehensive set of design actions
and good navigation and content reviewing capabilities.

\begin{figure}[!ht]
	\centering
	\includegraphics[width=12cm]{gfx/workflow1.png}
	\caption{Architectural project work flow}
	\label{FIG-WORKFLOW1}
\end{figure}

% remote and multi-user reviews - costs and cycle
The possibility to make part of a review session remotely can benefit both parts:
it is more cost-effective for the architects than hosting each review event and
it allows a shorter reviewing cycle, 

% merge with environment and experiment more
The architect can benefit from the system by being able to experiment different locations
for the building and obtain a better integration between the new design and its surroundings.

% constraints to the architect
In order for an architect to use a computer program for drawing early designs,
it must be flexible enough not to limit the architect in expressing his vision.
Most current systems constrain architect's creativity \cite{TOW3D}.

% constraints to the client



% multimodal int.
A set of input and output modalities must be assembled and orchestrated

 must then be assembled with the purpose of offering
the users an immersive experience. Therefore this set must be chosen and
means must be given to users to allow them to perform actions like walking or creating
annotations.

% navigation, multi user
Having also the purpose of client reviewing and showcasing of designs,
the system should offer a comprehensive interface,
one that aids the user in navigating the world with a gentle learning curve.
Reviewing could also be done remotely.
Either in a local or remote set, the system should allow
concurrent browsing and editing of the world.




\section{Existing Solutions}
% EXISTING SOLUTIONS
% seek existing solutions?


\section{Approaches}
\TODO{articles about viewing interface,
constraints
and getting user input,
add smart sketchpad image,
end Navigation}
% group articles by domain 
% for every article write a short summary of its relevant features
% and add an illustrative picture
% then criticize the presented solutions

\subsection{Viewing Interface}
In order to obtain an immersive experience, there's a number of hardware
setups commonly available:

\begin{description}
	\item[Head Mounted Display (HMD)] --
	  Head mounting displays are devices shaped like glasses, projecting a pair of stereo
	  transformed images to the user's retinas.
	  They normally feature a gyroscope or similar apparatus to measure head orientation and tilt.
	  There are two kinds of HMDs: in the former the images are projected in small opaque screens;
	  in the latter the projected surface is translucid, allowing blending of real and virtual worlds.
	  
		Using a head mounted display has the benefit of sticking to the user's head
		and detecting head orientation.
		On the other hand each HMD serves one single user.
		Additionally, most users report suffering from fatigue after long periods of
		usage and it has limited resolution.
			
	\item[Cave Automatic Virtual Environment (CAVE)] --
	  A CAVE is an immersive virtual reality environment where projectors are directed to four,
	  five or all the six walls of a room-sized cube.
	  
		It shares the benefit of enclosing the user's viewing area with HMDs.
		Has a better resolution though.
		The downside is the small number of simultaneous users who can experience the CAVE at the same time.
	
	\item[Wall] --
	  A wall is a large surface, usually planar, filled by an image.
	  The whole image projection is responsibility of a cluster of projectors set up in a wall.
	  Each projector renders part of the surface and the border between projections is ideally minimal.
	  Each projector is controlled by an independent computer.
	  
		Its size and resolution depend entirely on the setup, but normally a wall offers high resolution
		(depends on the number of projectors in the grid and each projector's resolution).
		Due to the large surface of the wall, several users may be served as once.
		The downside is users having to face the wall to experience the image entirely.
\end{description}

\begin{figure}[!ht]
	\centering
	\includegraphics[width=12cm]{gfx/hmd-cluster-cave.png}
	\caption{HMD, Wall, CAVE}
	\label{FIG-HMD-CLUSTER-CAVE}
\end{figure}

Any of these setups is suitable for single user interaction.
In case of a reviewing session, in which at least two participants are required,
CAVE or Wall are better suited, since they alone offer a solution for a small group.

Using a Wall or CAVE presents other challenges: the computers responsible for
generating each projectors' images must be synchronized, its' color parameters calibrated,
the viewport must be well cropped, etc.
Several systems exist capable of delivering high performance 3D graphics and
offering the features mentioned above.
Based on scene graphs there are two well known solutions: OpenSceneGraph and OpenSG.

\subsection{Representing Urban Scenery}

How to effectively render city landscape.
One has to limit the detail of objects further away.
Ideally the transition should be smooth but recognizing each building's main shape
even far away is equally relevant.
This can be achieved implementing a solution like the following.

\subsubsection{Continuous LOD}
J�rgen D�llner and Henrik Buchholz \cite{LODCITY05} present a
solution for modeling buildings that feature a continuous level of detail.

The authors propose the following levels of detail for a building (derived from
CityGML\footnote{CityGML is a common information model for the representation of 3D urban objects.
It defines the classes and relations for the most relevant topographic objects in cities
and regional models with respect to their geometrical, topological, semantical and appearance properties.
More info at: http://www.citygml.org}, see Figure \ref{FIG-LODCITY}):

\begin{itemize}
	\item Simple block model
	\item Model with defined roof geometry
	\item Detailed indoor and outdoor building features
\end{itemize}

\begin{figure}[!ht]
	\centering
	\includegraphics[width=15cm]{gfx/lodcity05-1.png}
	\caption{Continuous LOD: a) block model; b) building split into floors; c) geometry refinement; d) appearance refinement.}
	\label{FIG-LODCITY}
\end{figure}

A building is composed of a list of floor objects. A floor object refers to a floor prototype, which
contains the floor specification. This indirection allows a user to reuse floor prototypes in several
floors of the same building and allows rendering optimizations.

Each floor prototype is defined by its ground plan, which is one or more polygons that define the area
on which walls can be constructed. There can be inner loops in order to allow features like courtyards.
A ground plan supports thickness, useful for defining terraces.

On top of a ground plan one can place walls. A wall represents a vertical, planar polygon.
The default type of wall has no thickness, sufficient if a group of them form a closed surface
and can be only seen from the outside. Thick walls can also be added.
A wall can be lower or higher than its floor height, allowing balcony fronts and chimneys to be defined.

The higher ground plans can have roofs. The most common roof types are supported (hipped, gabled, tent,
mansard, pent, barrel) and a roof is described only by choosing the floor type and placing its
most relevant points (known as the roof skeleton).

Each floor prototype has a related floor decoration. A floor decoration is a collection of facade sections
and window sections. The former allows whole wall sections to be assigned a material while the latter
allows the definition of positioning and appearance of the floor's windows.


\subsection{Getting User Input}
Obtaining user input can be handled by an infinite combination of interfaces.
In virtual reality the most common solutions may feature a combination of:
\begin{description}
	\item[Image Processing] --
		A camera or a set of cameras continuously captures the environment and detects relevant
		features using vision algorithms.
		Subjects' position and orientation may then be determined using triangulation techniques.
	\item[Speech Recognition] --
		Users give specific verbal orders captured by a microphone, 
		which are interpreted by a speech recognition engine.
		The freedom of speech handled by the engine depends on its quality and the purpose of
		the recognition -- spanning from simple commands to free formed sentences.
	\item[Motion Tracking] --
		This method is akin to image processing, but optimized for tracking subjects position
		and orientation.
		Small reflective markers are attached to each subject relevant parts and infrared cameras
		are calibrated in order to be able to read the markers and determine their position
		and orientation precisely.
		See Figure \ref{FIG-MOTION-TRACKING}.
	\item[HMD's Rotation Data] --
		When a user is wearing an HMD device, its head rotation and tilt data can be captured,
		allowing the virtual world to act accordingly, e.g. rotating and tilting the VR world.
	\item[Tracked Artifacts for Direct Manipulation] --
		When motion tracking is available, users can manipulate artifacts and their positioning,
		orientation and/or relative status can determine actions in the world, e.g. point at
		something, dragging, rotation.
	\item[Space Ball, Space Pilot, etc.] --
		These are special devices which allow 6 degrees or freedom (6DOF) manipulation.
		The user holds the device in his hands and the device's data can map several actions to
		the read axes.
		See Figure \ref{FIG-SPACE-DEVICES}.
\end{description}

\begin{figure}[!ht]
	\centering
	\includegraphics[width=8cm]{gfx/MotionCapture.jpg}
	\caption{Motion tracking example:
		the subject wears a suit with several markers distributed so that his skeleton
		relative orientations can be reasoned.
		An array of infrared cameras is rigged to the ceiling, reading marker positions.}
	\label{FIG-MOTION-TRACKING}
\end{figure}

\begin{figure}[!ht]
	\centering
	\includegraphics[width=10cm]{gfx/space-devices.png}
	\caption{Two kinds of space balls and one space pilot.}
	\label{FIG-SPACE-DEVICES}
\end{figure}

The least intrusive interface would use speech recognition and either motion tracking or image processing,
the former being more driven to the purpose.
Speech recognition allows commands to be given to the system.
The remaining interface allows knowing positioning and rotation of users or parts of their bodies.
Tracking artifacts may either serve as a pointer or as a metaphor for direct manipulation.

\subsection{Sketching for Geometry Creation and Editing}

\subsubsection{Smart Sketchpad}
Wenyin et al. created Smart Sketchpad \cite{SMARTSK01}.
Smart Sketchpad recognizes standard shapes (rectangles, triangles, ellipses, straight lines)
and compound shapes such as arrowheads.

The article describes the steps necessary for shape recognition:
\begin{enumerate}
	\item Input as a chain of points;
	\item Polygonalize to polyline and refine endpoints;
	\item Close near endpoints. If closed go to step 6;
	\item If line ends near another line end, join them and go to step 3;
	\item Classify line as one of: straight line, polyline or free form curve. Go to step 8;
	\item Close shape recognition; *
	\item Estimate the parameter of the closed shape;
	\item Test if shape can be combined with other shapes in the drawing. If so repeat step 8, else end.
\end{enumerate}

The $6^{th}$ step was tested with both rule-based systems, support vector machines and neural networks.
The most successfully approach was SVN, with 97,5\% success, closely followed by NN.

\subsubsection{Assist}
Alvarado and Davis present a work towards an interface for mechanical designers named Assist \cite{FREEDOM01},
with the purpose of allowing them to sketch naturally and have the computer interpreting
their strokes into shapes like rods, hinges, polygons, etc.

The interpretation is a three stages procedure:

\begin{itemize}
	\item Match strokes to a series of templates;
	\item Rank interpretations with several heuristics about drawing style and mechanical engineering;
	\item Return the most consistent hypothesis.
\end{itemize}

Figure \ref{FIG-FREEDOM01} illustrates the result.

The authors emphasize the difficulty they faced when replacing the original
human-made strokes by its computer interpretations.
Users prefer composing the whole drawing prior to computer interpretation replacement,
as opposed to a changeable estimation that refreshes at every added stroke.
Having the computer interpreting every stroke makes them feel they're losing control of the program. Even so, that was the path chosen by the authors because every extra stroke without giving feedback to the user increases the chance of misinterpretations.

Another relevant conclusion is that users expect symmetry to be kept regarding the interpreted
shapes, so it would be a good idea to detect and suggest alignment restrictions between shapes.

\begin{figure}[!ht]
	\centering
	\includegraphics[width=15cm]{gfx/freedom01-1.png}
	\caption{Assist: A car on the hill. Drawn by the user (left), as interpreted and displayed (right).}
	\label{FIG-FREEDOM01}
\end{figure}

\subsubsection{Digital Clay}
Digital Clay \cite{DIGCLAY00} allows a user to draw freely and tries to convert the drawing into a 3D model.
See input and 3D result in Figure \ref{FIG-DIGCLAY}.

It uses two techniques to achieve that:

\begin{itemize}
	\item The Huffman-Clowes algorithm, which identifies concave and convex vertices,
	requiring every line to connect to another line.
	The program additionally demands the object drawn to be solid;
	\item The 3D coordinates are inferred based on inherent rules
	that govern each type of drawing projection.
	Examples of applied rules:
	\begin{itemize}
		\item Axes of isometric drawings have equal angles between them;
		\item Perspective drawings show foreshortening of lines as we get closer to the viewer.
	\end{itemize}
\end{itemize}

The downfalls of this method are
the impossibility of describing occluded faces and
the limited editing capacities available once the conversion has been done.

\begin{figure}[!ht]
	\centering
	\includegraphics[width=15cm]{gfx/digclay00-1.png}
	\caption{Digital Clay: Raw sketch input and its 3D interpretation.}
	\label{FIG-DIGCLAY}
\end{figure}

\subsubsection{Discussion}

In \cite{SMARTSK01} a useful comparison between alternative implementations of their
shape recognition algorithm is performed.
Wenyin at al. conclude in their paper that Support Vector Machines are the faster implementation.
The proposed algorithm itself is relatively simple and may be generalized to 3D shapes.

The method proposed by Alvarado and Davis in \cite{FREEDOM01} works nicely but has applications
only in 2D space.
Additionally, users felt uneasy as the geometric shapes they drawn kept continuously being
replaced by the computer program, making them feel having lost control of the application.

Schweikardt and Gross \cite{DIGCLAY00} developed the only system discussed here that allows 3D drawing.
Though its results are interesting, it shouldn't be used because it only allows describing part
of geometry (the occluded geometry is undefined) and because it doesn't allow successive
iterations to fill in the undefined geometry.

\subsection{Helping the User: Suggested Constraints}

\subsubsection{Pegasus}
Igarashi and Hinckley present a 2D sketching system named Pegasus \cite{BEAUTY97}.
It receives user strokes and converts them, generating candidates by taking into
account restrictions for:
vertex connection, segment connection, parallelism, perpendicularity,
alignment, congruence, symmetry and interval equality.

The application presents the most relevant candidates to the user and highlights
the highest relevant one.
The user can either accept it or select another candidate by tapping on it as seen
in Figure \ref{FIG-PEGASUS}.

\begin{figure}[!ht]
	\centering
	\includegraphics[width=6cm]{gfx/beauty97-1.png}
	\includegraphics[width=6.5cm]{gfx/beauty97-2.png}
	\caption{Pegasus: A diagram drawn on Pegasus without using any editing commands such as rotation,
		copy or griding (left); interaction with multiple candidates (right).}
	\label{FIG-PEGASUS}
\end{figure}

\subsubsection{Discussion}

Igarashi at al. present in \cite{BEAUTY97} a useful way of aiding the user.
Most elements in architecture subsume to these constraints.
Applying a technique of this kind gains more relevance in
the 3D architecture problem because users will be modeling in 3D.
Humans are more prone to drawing errors in a task like this,
benefiting more from beautification algorithms.

\subsection{Navigation}

\subsubsection{Smart and Physically Based Navigation}
In order to ensure users not ``getting lost'' in the virtual space,
Buchholz, Bohnet and D�llner \cite{SMARTCAM05}
propose a camera that is is both smart and physically-based.
It is smart in the sense that it is aware of confusing,
disorienting viewing situations, providing means to circumvent them.
It is physically based because it is supported by a physically based model of 3D motion
to ensure steady, continuous user movements.

In order to solve the disorientation problem, the camera must identify situations
when to intervene.
For that matter a metric, called orientation value, was created.
Each view is classified by counting its pixels, granting each one a different value:
landmarks are granted the highest values; terrain gets high values and sky gets low values.
A threshold can then be established and views below it are classified ``disoriented''.

When such an event takes place, smart navigation techniques restrict camera control.
The constraints posed to user control must be as comprehensible as possible.
Camera movement should also be time-coherent and physically sound.

The maintenance strategy solves critical situations such as (see Figure \ref{FIG-SMARTCAM1}, left):

\begin{enumerate}[a)]
	\item The user rotates the flight direction and causes the camera to look too far beyond the terrain border.
		The rotation is accepted but outweighed by a slight rear movement away from the border.
	\item The user is flying forward beyond the terrain border.
		The maintenance strategy temporarily tilts down the view direction until a maximum angle is reached.
	\item If no more tilting is possible, the strategy rotates the flight direction parallel to the terrain
		to fly along the terrain border.
\end{enumerate}

\begin{figure}[!ht]
	\centering
	\includegraphics[height=8cm]{gfx/smartcam05-1.png}
	\includegraphics[height=6cm]{gfx/smartcam05-2.png}
	\caption{SPB Cam: Maintenance strategy for keeping high orientation values (left); Adjustment strategy for ensuring of feasible view specifications (right).}
	\label{FIG-SMARTCAM1}
\end{figure}


%\subsubsection{}

%\cite{SKAN02}
%\cite{DESFUT04}



\section{Comparative Analysis}
\TODO{add my ideas for solution}
% comparison table of various solutions
% Analyze and comment solutions
% Add out idea for solution

% summary of section:
In this section we will analyze popular solutions available in the market,
highlighting each solution strenghts and weaknesses.
Later on a table is presented summarizing each solution and relevant features compared, and its
data discused.
The section ends with the author's ideas for an alternative solution.

\subsection{AutoCAD}
AutoCAD is the \emph{de facto} standard software for architecture designs.
It has a steep learning curve (Figure \ref{FIG-AUTOCAD}) but is nevertheless
learned all over the world.

One of its formats, DXF, is widely supported in 3D modelers and engines.
AutoCAD features a powerful language, AutoLisp, which allows advanced users to create scripts for
automating any aspect available in the interface.

This program favors 2D drawing and modeling over real 3D concepts,
but a skillful operator can create every shape necessary to an architectural scenario.
Internally AutoCAD doesn't support any interactive preview of the created designs.

It renders using the Mental Ray\footnote{Mental Ray rendering engine -- http://www.autodesk.com/mentalray}\nocite{SITE-MENTAL} engine.
Animations can be made using camera paths.

\begin{figure}[!ht]
    \centering
    \includegraphics[width=12cm]{gfx/autocad-1.png}
    \caption{AutoCAD: A typical layout of a floor plan.}
    \label{FIG-AUTOCAD}
\end{figure}

\subsection{ArchiCAD}
\nocite{SITE-ARCHICAD}
GraphiSoft ArchiCAD\footnote{GraphiSoft ArchiCAD -- http://www.graphisoft.com/products/archicad}
aims at conquering new architecture students who haven't been exposed to AutoCAD.

It offers pre-made views and document templates for every architectural driven need.
It is by definition a 3D CAD program and it is praised by architects for its easy 3D
manipulation capabilities (Figure \ref{FIG-ARCHICAD}).

The program features templates for common architectural elements.
ArchiCAD has navigation capabilities too, allowing first person perspective navigation of the model.
Its workflow is thought out to make it easy for an architect to do the most common tasks,
making it a friendlier alternative when compared to AutoCAD.
It lacks the expressive power to do about 10\% uncommon tasks though.

There's a Software Development Kit for ArchiCAD plugin creation.

\begin{figure}[!ht]
    \centering
    \includegraphics[width=12cm]{gfx/archicad-1.png}
    \caption{ArchiCAD: Notice basic selection and template properties editing in the 3D view.}
    \label{FIG-ARCHICAD}
\end{figure}

\subsection{Revit Building}
\nocite{SITE-REVIT}
Revit Building\footnote{AutoCAD Revit Building -- http://www.autodesk.com/revit}
is another Autodesk product.
Unlike AutoCAD, which spans its use to other areas such as mechanical engineering,
Revit Building was explicitly thought out for architectural design.

It works completely in 3D and has native templates for doors, windows, roofs, etc.
Features the concept of mass, lacking in most packages.
Common constraints are detected. Thick walls can be drawn as lines and solids can
be cut as floors (compare the top-left and bottom right views in Figure \ref{FIG-REVIT}).

Revit Building features powerful templates for complex tasks such as roof design.
It has a simple raytacer and radiosity engine. Cameras can be placed but only for view
rendering, not animation.

Being a system for the professional segment, it has a lower learning curve and provides
tools that allow successful modeling of complex buildings, even for enthusiasts,
something much harder to do in AutoCAD.

\begin{figure}[!ht]
    \centering
    \includegraphics[width=12cm]{gfx/revit-1.png}
    \caption{Revit Building: Solid shapes can be combined with boolean operations and converted into buildings.}
    \label{FIG-REVIT}
\end{figure}

\subsection{SketchUp}
\nocite{SITE-SKETCHUP}
Google SketchUp\footnote{Google SketchUp -- http://www.sketchup.com}
is a friendly program for the novice 3D modeler.
It features a simple interface and most of its tools are basic,
still able to achieve acceptable results.
Its learning curve is great -- everyone can sketch a room in a nick of time.

Its engine is based on drawing lines on top of lines,
already created surfaces or a construction plane.
It detects the most common geometry restrictions (such as midpoint and perpendicularity).

Features an online repository of models, allowing importing of objects such as
furniture, trees, props or well known buildings by just browsing and selection.

Performs strange results when handling awkward angles or when several lines
are the vicinity of the mouse.
Curve manipulation and generation of surfaces is nonexistent.
Allows plugin design in the Ruby language.

Another bonus from being part of the Google software library, SketchUp features import/export capabilities to Google Earth
\footnote{Google Earth -- http://earth.google.com}\nocite{SITE-EARTH}.
This allows capturing a patch of land to SketchUp, design a building there and export the patch
back with its new contents to Google Earth.

A great feature SketchUp is realtime shadows (Figure \ref{FIG-SKETCHUP}) -- since there's only one viewport, shadows are crucial to give the user a sense of depth to a scene.

In renders cartoonish styled views and allows interpolation of cameras to render simple animations.

The professional version of the program allows exporting to common 3D architectural formats.


\begin{figure}[!ht]
    \centering
    \includegraphics[width=12cm]{gfx/sketchup-1.png}
    \caption{SketchUp: Basic shape extrusions. Notice the shadows in the viewport.}
    \label{FIG-SKETCHUP}
\end{figure}

\newpage

\subsection{Comparison Table of Available Solutions}
\begin{table}[!ht]
    \centering
		\begin{tabular}{|c|c|c|c|c|}
			\hline
			\backslashbox{Features}{Solutions}		& AutoCAD		& ArchiCAD	& Revit Building	& SketchUp	\\
			\hline
			Design in 2D						&		\GdA		&		\GdB		&				\GdB			&		\GdC		\\
			\hline
			Design in 3D						&		\GdD		&		\GdC		&				\GdB			&		\GdB		\\
			\hline
			Architectural Templates	&		\GdD		&		\GdB		&				\GdA			&		\GdC		\\
			\hline
			Supported Formats				&		\GdC		&		\GdC		&				\GdB			&		\GdD / \GdB \footnotemark\\
			\hline
			Interactive Modes				&		\GdE		&		\GdB		&				\GdC 			&		\GdE		\\
			\hline
			Rendering Capabilities	&		\GdB		&		\GdB		&				\GdC			&		\GdD		\\
			\hline
			Extensibility						&		\GdA		&		\GdC		&				\GdE			&		\GdC		\\
			\hline
		\end{tabular}
		\caption{Different solutions in the market compared.}
		\label{TB-COMP-SOL}
\end{table}
\footnotetext{SketchUp exporting capabilities depend on using free or commercial version}

\begin{table}[!ht]
    \centering
		\begin{tabular}{|p{2cm}|p{2cm}|p{2cm}|p{2cm}|p{2cm}|}
			\hline
			very bad	& bad			& average	& good		& very good	\\
			\hline
				\GdE		&	\GdD		&	\GdC		&	\GdB		&	\GdA			\\
			\hline
		\end{tabular}
  \caption{Legend of Table \ref{TB-COMP-SOL}}
  \label{TB-COMP-SOL-LEGEND}
\end{table}

\subsection{Discussing the Table}
AutoCAD makes use of a well established workflow, which takes time to master.
There's a way of doing everything architecturally speaking, though most times a difficult one.
Since its a package general enough for supporting other uses, AutoCAD doesn't come with architectural
templates, a very helpful feature available in both ArchiCAD and Revit Building.

Revit Building is Autodesk's vision of an easy to master, yet powerful system for architectural design.
Revit Building and ArchiCAD are the most similar systems.
Revit Building has better modeling features while ArchiCAD has many document templates
ready for extracting bureaucratic papers out of the architect's workflow.

SketchUp is undoubtfully the most amateur of the analyzed systems.
It offers limited geometrical operations and doesn't have a real template library.
It tries to overcome that limitation by offering a large online repository of models.
SketchUp's best qualities are its learning curve -- any user can feel confident in
modeling a simple floor in a couple of minutes -- and the Google Earth connection.

It would be of great use if other programs were granted permission to get geographical
data (both height maps and texture maps) of an area on Earth.
This offers an important head start for an architect in designing a building that
smoothly blends in its surroundings.


\section{Conclusions}
\TODO{Conclusions, Future Work}
% a summary of the most relevant conclusions that appear throught the document

% directions where to lead efforts in future work


\bibliographystyle{splncs}
\bibliography{thesis}

\end{document}
