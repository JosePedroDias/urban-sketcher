\section{Conclus�es}

%revista de forma a mostrar mais valia da abordagem
%com feedback dos utilizadores,
%e comparar a experiencia tradicional de navega��o, n�o existe so
%limita��es technologicas,
%a abordagem � alternativa e multimodal, referir se � natural ou viavel ?
%Ou seja a avalia��o deve ir de alguma forma ao encontro do que �
%referido no abstract e conclus�o.


Neste artigo apresent�mos uma interface multimodal simples mas poderosa
para suportar as ac��es mais comuns de manipula��o e visualiza��o tridimensional
em ecr�s de larga escala, combinando captura de movimentos, fala e gestos numa
interface natural.

Testes realizados com utilizadores permitiram verificar a efic�cia
e virtudes da nossa abordagem m�os-livres que substitui com vantagem
t�cnicas mais convencionais.


%O objectivo foi oferecer aos utilizadores
%uma interface simples mas suficientemente poderosa para
%capacit�-los nas ac��es mais comuns de manipula��o tridimensional.
%Julga-se t�-lo conseguido com uma maior imers�o no ambiente virtual.



%as opera��es permitem ao utilizador uma maior imers�o no ambiente
%virtual, j� que n�o o for�am a interagir com os tradicionais menus, 
%que levam a uma mudan�a de contexto.

%Notou-se um elevado n�vel de entusiasmo por parte dos utilizadores
%que testaram o nosso sistema, particularmente no modo de voo.
