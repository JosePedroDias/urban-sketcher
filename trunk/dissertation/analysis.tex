\chapter{Analysis}

\section{Comparative Analysis of Building Modeling Software}

\TODO{use the comparative analysis from survey}

\section{Common Modeling Operations Offered by Polygon-based 3D Software}

\TODO{inspect modeling operations from common modeling software}

\section{Interviews with Architects}

The author conducted two interviews with experts in this domain.
The interview with Architect Jos� Seco focused mainly on ways to classify
buildings. The second interview with \TODO{NAME?} was conducted with
the main concern of understanding the architect current work flow and 
ways to improve it.

\subsection{Building Characteristics}

A building can be described in several dimensions.

Its purpose: residential buildings, office, retail commerce, factory facilities, schools, hospitals, leisure, etc.

Its volume: defined by the number of allowed floors.
This comes from a maximum floor height, defined by the local \TODO{PODER}, based on the population density
set to the area of construction.

Blueprint Configuration: influence by the topography, terrain type, access routes (roads, etc).

A building may have different floors serving different purposes,
ex: first floor with small commerce and remaining ones for residential purposes.

Another legislated subject is the useful construction area ratio, defining the terrain left empty, green areas, etc.

The ceiling of a building can be classified by either being terrace-based (plain) or being slanted.

There are three common relative distribution of buildings:
\begin{itemize}
	\item individual buildings, with a large area separating each one;
	\item \TODO{GEMINADOS}, with one wall shared by each pair of consecutive buildings;
	\item \TODO{EM BANDA}, with a set of buildings lined up in a direction, with a minimum distance between them.
\end{itemize}

\TODO{FIGURE SHOWING RELATIVE DIST OF BUILDINGS}

There are additional issues regarding building planning.
Most materials should ideally come from nearby natural resources;
the glassed areas in the facades respect a ration (ex: $\frac{1}{10}$ of the facade).


\subsection{Work Flow Improvements}

Nowadays most \TODO{CAMARAS?} can provide digitally accurate data of the area the architectural
project is destined to occupy. The most common and useful documents are
aerial photographies of the terrain
\TODO{CURVAS DE N�VEL}, providing discrete but accurate measuring of the area topography.

Making use of the provided data early in the project would be very effective.
Most architects and engineers need to set the exact place where the project is to be developed
by reading the hard to interpret \TODO{CURVAS DE N�VEL} and/or by effectively scouting the
area on foot.

An alternative would be to feed the system with this data and get a simulation of the actual 3D
area, both in terms of appearance and volume. If the system could simulate lighting conditions
-- i.e. Sun at different day times and different months --  for lighting studies,
with early sketch studies of the main volumes tried at different locations in the area.
A good sectioning tool would be of great use in this task too, allowing a more effective reading
of the interaction of building and terrain at different angles.

The development of the main building volumes is an iterative process with strong subjective aspects.
In can begin with box shapes making up the volumes, testing the set with different lighting conditions.
After being asked for a sketch-based approach for the definition of the building volume,
the proposed process started from a 2D blueprint being extruded upward,
followed by a redefinition of edges and curves.

\TODO{FIGURE SECTION, DIFFERENT LIGHTING CONDITIONS, REDEFINITION OF VOLUMES}

If on site visits are made, pictures or video footage can be captured and their content could be
attached to their virtual counterparts, so that by exploring the virtual representation of the construction
area one could make use of real views captured earlier.
This would allow other members of the team to get a more integrated presentation of this data
and for a straightforward way of later referencing them by inspection of the virtual scene.

\subsection{Discussion}

From the several discussed building properties, some of them must be set by the user
while others could be derived from related data.
Strong candidates for properties set by user intervention are both the
buildings blueprints and the building's height.

The placement of the buildings should be free but means should be given so that
making blocks of similar buildings to be a simpler process,
supporting the most common relative building distributions.

The system should facilitate the fast creation of building facades.
The architectural styles should map the most common building purposes and styles, so that the creation
of a large building (or several of them) can be a matter of defining placement, volume and purpose of a building.
The materials, floor heights and roof types of buildings could also be mapped into the architectural styles' properties.

Regarding the creation of the surrounding environment where the project is to take place,
a more tight integration between provided content and the projected architecture could be accomplished.

Even if not directly supported by the system, the usage of height maps / \TODO{CURVAS DE N�VEL} and aerial photography
for the generation of the virtual scenario where the building creation and experimenting is to take place should
be streamlined with a work flow.

Currently the software used by architects does not provide out of the box support
for the integration of such data, let alone the correct exploitation of its potential,
therefore a well thought out navigation system should be provided so that a creative user could easily find areas for
construction and a client user could easily explore the virtual scenario.