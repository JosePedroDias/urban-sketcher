\abstract{
Este artigo apresenta o trabalho desenvolvido na procura de modos alternativos de navega��o e manipula��o geom�trica frente a ecr�s de larga escala.
Na tela em frente ao utilizador � projectado um mundo virtual na primeira pessoa.
O utilizador equipa-se com um conjunto de marcadores reflectores nos bra�os e o seu movimento � capturado por um conjunto de c�maras que operam na gama dos infra-vermelhos e identificam com precis�o a localiza��o de cada marcador no espa�o em frente �tela.
Mapeou-se a interac��o de voar sobre a cena erguendo os bra�os para mimificar o voo do Super-Homem.
O reposicionamento de objectos faz-se sugerindo a direc��o de deslocamento com o bra�o.
A rota��o de objectos faz-se rodando num plano imagin�rio frente ao utilizador, paralelo � tela.
A altera��o de escala faz-se abrindo ou fechando os bra�os de forma a sugerir a altera��o a imprimir no objecto.
De modo a detectar inequivocamente cada modo sem necessitar de manusear dispositivos adicionais, a sua escolha efectua-se ditando comandos
de fala.
Foram identificados trabalhos relevantes desenvolvido por outros, sendo descrita a solu��o encontrada, os problemas que procura minorar e os testes conduzidos com utilizadores.
Termina-se com um conjunto de ideias para prosseguir este trabalho.
}

\keywords{detec��o de movimentos, manipula��o directa, navega��o, interac��o por via de gestos.}

