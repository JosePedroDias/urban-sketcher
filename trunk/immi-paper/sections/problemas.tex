\section{Problemas}

\subsection{Motion Tracking}

A configura��o ARTrack tem apenas duas c�maras frontais ao utilizador e ambas t�m de captar os pontos de refer�ncia usados nos bra�os do utilizador. N�o havendo redund�ncia de c�maras � relativamente frequente a oclus�o dos pontos de refer�ncia quando se utilizam os bra�os entusiasticamente. O problema foi resolvido recorrendo o outro sistema de motion tracking Matrox, desta vez com quatro c�maras � volta do utilizador. A fiabilidade aumentou substancialmente e permite ao utilizador uma maior liberdade de movimentos.

N�o foi utilizada a fase final da plataforma de tracking disponibilizada (onde � calculada a matriz transforma��o por artefacto), neste caso espec�fico uma matriz por cada bra�o do utilizador. Foi tomada tal decis�o pois as matrizes obtidas falhavam frequentemente o sentido da orienta��o e eram pouco precisas, al�m do facto da sua obten��o n�o ser uma mais-valia como dado de entrada no sistema.

\subsection{Voz}

Foi necess�ria a altera��o da gram�tica escolhida para desambiguar alguns comandos de voz. As palavras "start", "stop" e "quit" s�o dif�ceis de distinguir, sendo erroneamente reconhecidas. Sem esta altera��o o sistema n�o seria us�vel.
