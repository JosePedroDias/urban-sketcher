% how and why to divide into these subproblems

% deliver?
We want to deliver a solution capable of being:
\begin{itemize}
	\item a sketch board for architects, who can create and edit buildings and their features
	\item a virtual urban scenery for anyone -- therefore with gentle learning curve -- and suitable for showcasing and reviewing architectural designs
\end{itemize}
In order to achieve these goals, there's a number of problems that need to be addressed.

% navigate in it?
% where -> at which?
One must be able to understand the world representation, navigate it and create or modify its contents.
In order to be immersed in the world, the interface should be more than a regular computer screen.
User input has to be based on devices that give him more expressiveness and allow human error,
preferably helping him out in subjects where a machine works better than humans do, such as figuring
out parallels or calculating areas.


% opensg/goggles/wall, etc
% city representations
\subsection{Representing Urban Scenery}
How can the system produce the most immersive experience for one user?
What if its a group of people? All this constrained by having to get user input at the same time.

How can we address the complexity of rendering a city?


% laser, motion, speech, gestures, smart widgets
\subsection{Getting User Input}
What are the viable alternatives to keyboard/mouse interaction with these objectives?
How can a user sketch his ideas, give orders, navigate?


% sketching
\subsection{Sketching for Geometry Creation and Editing}
Given sketching input, how can the system interpret the sketch and reason an object
out of it?
 

% by which?
% Helping the User: 
\subsection{Helping the User: Suggested Constraints}
There's a subset of geometry that appears quite often in buildings.
The walls are generally box shaped, as are most windows. We often use concepts of symmetry,
the golden ratio, cylinders, etc.

It would be of great use if while sketching one could be aided in keeping the ratio of
a rectangle, draw parallel or perpendicular lines, do extrusions and lathes.

What are the available ways by which one could be aided in these tasks?


% navigation
\subsection{Navigation}
By what means should the user convey his position and rotation in the world? What can he
do and how? How to avoid users getting lost in the virtual world?
