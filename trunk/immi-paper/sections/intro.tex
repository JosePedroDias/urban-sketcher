\section{Introdu��o}\label{sec:introducao}
A navega��o numa cena tridimensional processa-se normalmente num cen�rio f�sico de computador de secret�ria, utilizando o rato e o teclado como dispositivos de entrada e um monitor como dispositivo de sa�da. De forma a tornar a navega��o no espa�o e outras actividades em experi�ncias imersivas foi desenvolvido no laborat�rio Prof. Louren�o Fernandes\cite{leme} um conjunto de modos de interac��o que se t�m provado eficazes para o efeito e recorrem a equipamento de captura de movimentos e um ecr� de larga escala para navegar no ambiente e manipular objectos na perspectiva da primeira pessoa.

O trabalho pretende proporcionar uma interac��o agrad�vel e mais directa para navega��o tridimensional, sem recurso a dispositivos intrusivos. Recorre a comandos de voz para seleccionar a ac��o a executar (navegar no espa�o, manipular objectos) e a movimentos dos bra�os para efectuar a ac��o.