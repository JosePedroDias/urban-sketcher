\chapter{Conclusion}

%%%%%%%%%%%%%%%%%%%

% motivation

The produced work was made to provide architects a way
of exploring large scale displays for the creation of urban sceneries.
This task can be performed collaboratively using common laser pointers.
The implementation of a set of navigational modes allow for easy
exploration of complex 3D scenes.
This scenario allows showcasing and reviewing of projects with clients.

The support for tablet PCs permits a portable and cost-effective alternative
to the large display screen.

A building creation method was developed that provides a fast way
of generating buildings based on their facade blueprints and height and
setting its purpose, with the system generating walls and facades that
reflect the choices made.


% aplicabilidade

\section{Main Contributions}

% work flow
This system was thought out to aid the task of generating 3D buildings.
An alternative work flow was defined where digital content supplied by
the municipal government or other entity is put to use -- a virtual 3D
version of the terrain is the starting point for the projection of
architectural buildings.
With it users have a way of generating buildings using a set of
architecture styles to generate its facades, wall materials and roof,
and they can do it surrounded by a simulation of the environment.

% u s concepts
For this task several concepts were introduced -- facade attachments,
template shapes, custom shapes and architecture styles.
Facade attachments can be generated from a 3D modeler such as Blender,
using an exporter plug-in to export the content into the internal shape format.
Architecture styles can also be created by users, providing a 
powerful way to define ``recipes'' for building creation, by setting the
odds of a certain type of door, window, balcony etc appearing in the facade,
setting floor height and layout, facade wall color and texture and roof type.

% custom shapes
In cases where custom 3D shapes are needed to fulfill a particular aspect
of a building, the user has the ability to create custom shapes.
These can be edited by a small set of face and edge operations,
studied to cover the most common needs of geometry modeling.
The most useful directions are available for the user to manipulate the
faces and edges, without the overhead of dozens of menu commands and parameters
so common in nowadays 3D modeling software.

% gates e menus
In order to solve the limitations of using laser pointers / tablet PC pens
for interaction, the gate concept was developed and exploited.
It provides a way of activating options by drawing strokes instead of clicking.
Stroke continuity data was used: the set of events from the start of the stroke
until its ending, passing through gates can trigger complex operations
such as dropping a shape from a menu into the 3D view.
A non-intrusive menu form was defined and applied, with a minimalistic approach
at the starting interface, letting the users invoke and position menus as they see fit.

% navigation modes
A set of navigational modes was developed to cover the most common exploration
and inspection tasks.
The common walk / fly mode was made available to give users the
possibility of exploring the scene as real users would. 
The compass navigation mode allows for seeing a top-down view of the nearby map,
allowing dragging the map for movement and  rotating the orientation
via a ring showing the cardinal points.
The examine navigation mode provides an effective way of inspecting an object of any
scale by dollying around it. The examined object or surface can be easily changed.
Having a motion tracking system installed, one can also fly using the arms.
This has proven both pleasing and effective, empowering the user of an alternative
way to rotate, change the altitude and flight speed by making use of arm movements.

%%%%%%%%%%%%%%%%%%%

\section{Future Work}

There is space for improvement in several aspects of the project,
even more since it addresses so many areas.

% work flow
The integration of real terrain data into the system could be eased by
the direct support of height maps, topography maps data such as contour lines
and aerial photographies. A work flow for the importing of such content is
even so outlined.

The facade walls could have forms other than rectangles and most work was
done so that such change could be easily supported, as long as a good
interface is provided to input such blueprints.

% realism
Lighting conditions could be manipulated inside the program with an
accurate simulation of daylight and shadowing, a subject highly valued
by architects for the definition of the location where a building is to be built.

Textures and materials could make usage of the latest developments in GPU
technology to simulate metals, glass and other materials and
techniques such as bump maps to give greater depth to facade details and textures.

% modeling operations
A more elaborate set of operations could be supported for modeling custom shapes.
Most operations could be generalized for application to a group of edges or faces.
Even so, the available operations are a good compromise between the complexity
of professional modelers and the stroke-based, minimalistic interface provided.

% voice
%Voice operations could 
