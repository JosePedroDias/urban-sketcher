% Contexto + Problema

% http://en.wikipedia.org/wiki/Architecture
Architecture is the art and science of designing buildings and structures.
It is an interdisciplinary field and has similarities to applied science and
engineering, but unlike them, which focus more on functional and feasibility aspects of design, 
architecture deals with building costs, space and volume, texture, light and shadow
in order to achieve an aesthetically pleasing result.

For centuries methods and norms have arisen. Architects tools of work were based on paper and ruler.
With the evolution of computational power throughout the last century, 
increasingly more complete, fast and robust Computer-Aided systems (CAD) were developed and so did
devices capable of manipulating architectural entities -- such as mouse, trackball, tablet, etc.

% only geographical, drawings?
The process of elaborating architectural designs usually starts by drawing rough sketches of the subject
in order to convey form, proportion and lighting. Relevant geographical data about the location
where the building is to be established is acquired. A series of two dimensional drawings is created
in order to precisely define features and dimensions of the building. These drawings serve as
input for civil engineers and the rest of staff responsible for constructing the building.
Three dimensional drawings and brochures are created and scale models are built to allow
people from other domains to better perceive the building before it is built.

Nowadays architecture is slowly but steadily embracing usage of computers in the designing process.
The elaboration of a set of views and floor plans necessary to construct a building is currently
subject to authoring in popular systems like Autodesk's AutoCAD\footnote{AutoCAD -- http://www.autodesk.com/autocad}.
Such systems, although allowing 3D volumes and surfaces to be built, were designed with the intent of
producing floor plans and standard views and therefore are optimized for those tasks.

% more references to software?
The creation of 3D drawings and brochures for depicting an intended reality is common practice,
relying in techniques such as
ray tracing\footnote{POVRay -- http://www.povray.org},
radiosity,
physically accurate light simulation\footnote{Maxwell Render -- http://www.maxwellrender.com; Indigo Renderer -- http://www.indigorenderer.com},
high dynamic range imaging (HDRI), etc.
These techniques produce believable results but take too much time to render in real time.

% frases enormes!
Another application of data coming from CAD systems is the generation of worlds optimized for navigation.
Performance is paramount in these systems so algorithms such as binary space partition (BSP) or octrees are applied
for better management of geometry rendering. The compromise between believability and fast rendering speeds
is assured with techniques such as light maps and good usage of graphics card's graphics processing unit (GPU),
allowing the application of many computer graphics algorithms that would otherwise be impossible with such time constraints.
There are numerouse 3D engines capable of presenting such worlds with good performance.
The major drawback is in navigating the scene. Most systems use popular metaphors of first or third person shooters and
their interface design (4 directions using keyboard plus rotation with a mouse). This solution works for youth
game enthusiasts but presents an obstacle for most people. There are other navigation metaphors available such as flying
but they keep using keyboard and mouse as their interface.

Exporting the geometry from a CAD program, which is a common solution, comes with several problems,
as stated by Alberto Raposo et al.\cite{CADVR06}:
\begin{itemize}
	\item \textbf{low performance} -- due to unneeded model complexity
	\item \textbf{lack of realism} -- usually users of CAD programs don't associate material
	or texture data to objects
	\item \textbf{inadequate treatment of geometry} -- the conversion often occurs with loss of
	geometry, precision and errors such as badly oriented normals
\end{itemize}

Using a computer system to present a virtual building with the purpose of selling an idea, doing real estate business
or simulating virtual tours demands better metaphors, better interface design and the least possible cluttering
of the view in order to achieve a rich experience and get an overall good impression.

From the architect's point of view, using such a system early in the process, if preceded by geographical data capturing,
allows him to better perceive surroundings where the building is to be constructed.
Having a system able to create 3D architecture on-sight as described could generate models able to
complement or even replace the first prototyping stages of the architectural design process,
commonly represented nowadays by rough sketches.

% Estrutura do documento - +/-
% later on?
We start by identifying the various domains in which problems arise. Then a series of approaches to solve them are
listed and we reason which better fit the problem at hand. Later on a comparative analysis between applications
in the market which allow architecture modeling is conducted. The document ends with a set of conclusions and
points out directions for future work in addressing this problem.
